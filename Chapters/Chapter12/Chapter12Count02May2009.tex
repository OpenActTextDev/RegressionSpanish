\setcounter{chapter}{11}

\chapter{Count Dependent Variables}


{\small \textit{Chapter Preview}. In this chapter, the dependent
variable $y$ is a count, taking on values 0, 1, 2 and so on, that
describes a number of events. Count dependent variables form the
basis of actuarial models of claims \emph{frequency}. In other
applications, a count dependent variable may be the number of
accidents, the number of people retiring or the number of firms
becoming insolvent.}

{\small The chapter introduces Poisson regression, a model that
includes explanatory variables with a Poisson distribution for
counts. This fundamental model handles many datasets of interest to
actuaries. However, with the Poisson distribution, the mean equals
the variance, a limitation suggesting the need for more general
distributions such as the negative binomial. Even the two parameter
negative binomial can fail to capture some important features,
motivating the need for even more complex models such as the
``zero-inflated'' and latent variable models introduced in this
chapter.}


\section{Poisson Regression}\index{regression model!count!Poisson}

\subsection{Poisson Distribution}\index{distributions!Poisson}

A count random variable $y$ is one that has outcomes on the
non-negative integers, $j=0,1,2,...$ The Poisson is a fundamental
distribution used for counts that has probability mass function

\begin{equation}\label{E12:PoissonDist}
\Pr \left( y=j\right) =\frac{\mu^j}{j!}e^{-\mu },~~~j=0,1,2,...
\end{equation}
It can be shown that $\mathrm{E~} y =\sum\nolimits_{j=0}^{\infty
}j\Pr \left( y=j\right) =\mu $, so we may interpret the parameter
$\mu $ to be the mean of the distribution. Similarly, one can show
that $\mathrm{Var~}y =\mu $, so that the mean equals the variance
for this distribution.

An early application (Bortkiewicz, 1898) was based on using the
Poisson distribution to represent the annual number of deaths in the
Prussian army due to ``mule kicks.'' The distribution is still
widely used as a model of the number of accidents, such as injuries
in an industrial environment (for workers' compensation coverage)
and property damages in automobile insurance.

\linejed\empexjed{SingaporeAuto}\index{datasets!Singapore automobile
data}\ecaptionjed{Singapore Automobile Data}

\textbf{Example: Singapore Automobile Data.} These data are from a
1993 portfolio of $n=7,483$ automobile insurance policies from a
major insurance company in Singapore. The data will be described
further in Section \ref{S12:SingaporeData}. Table \ref{T12:Table121}
provides the distribution of the number of accidents. The dependent
variable is the number of automobile accidents per policyholder. For
this dataset, it turns out that the maximum number of accidents in a
year was three. There were on average $\overline{y}=0.06989$
accidents per person.

\begin{table}[h]
\caption{\label{T12:Table121} Comparison of Observed to Fitted
Counts Based on Singapore Automobile Data}
\begin{center}
\begin{tabular}{crr}
\hline
Count & Observed & Fitted Counts using the \\
$(j)$ & $(n_j)$ & Poisson Distribution $(n\widehat{p}_j)$ \\
\hline
0 & 6,996 & 6,977.86 \\
1 & 455 & 487.70 \\
2 & 28 & 17.04 \\
3 & 4 & 0.40 \\
4 & 0 & 0.01 \\ \hline
Total & 7,483 & 7,483.00 \\ \hline
\end{tabular}\end{center}

\linetjed
\end{table}


Table \ref{T12:Table121}\ also provides fitted counts that were
computed using the maximum likelihood estimator of $\mu$.
Specifically, from equation (\ref{E12:PoissonDist}) we can write the
mass function as $\mathrm{f}(y,\mu) = \mu^y e^{-\mu} /y!,$ and so
the log-likelihood is
\begin{equation}\label{E12:BasicLogLike}
L(\mu) = \sum_{i=1}^{n} \ln \mathrm{f}(y_i,\mu) =
\sum_{i=1}^{n}\left( -\mu +y_i\ln \mu -\ln y_i!\right) .
\end{equation}
It is straight-forward to show that the log-likelihood has a maximum
at $\widehat{\mu }=\overline{y}$, the average claims count.
Estimated probabilities, using equation (\ref{E12:PoissonDist})\ and
$\widehat{\mu }= \overline{y}$, are denoted as $\widehat{p}_j$. We
used these estimated probabilities in Table \ref{T12:Table121} when
computing the fitted counts with $n=7,483$.

\index{goodness of fit statistics!Pearson chi-square}

To compare observed and fitted counts, a widely used goodness of fit
statistic is \emph{Pearson's chi-square statistic}, given by
\begin{equation}\label{E12:Pearson}
\sum_j\frac{\left( n_j-n\widehat{p}_j\right)^2}{n\widehat{p}_j}.
\end{equation}
Under the null hypothesis that the Poisson distribution is a correct
model, this statistic has a large sample chi-square distribution
where the degrees of freedom is the number of cells minus one minus
the number of estimated parameters. For the Singapore data in Table
\ref{T12:Table121} , this is $df=5-1-1=3$. It turns out the
statistic is 41.98, indicating that this basic Poisson model is
inadequate.\index{distributions!chi-square}

\subsection{Regression Model}

\index{exposure}\index{symbols!$E_i$, exposure}

To extend the basic Poisson model, we first allow the mean to vary
by a known amount called an \emph{exposure} $E_i$ , so that
\begin{equation*}
\mathrm{E~}y_i=E_i\times \mu .
\end{equation*}
To motivate this specification, recall that sums of independent
Poisson random variables also have a Poisson distribution so that it
is sensible to think of exposures as large positive numbers. Thus,
it is common to model the number of accidents per thousand vehicles
or the number of homicides per million population. Further, we also
consider instances where the units of exposure may be fractions. To
illustrate, for our Singapore data, $E_i$ will represent the
fraction of the year that a policyholder had insurance coverage. The
logic behind this is that the expected number of accidents is
directly proportional to the length of coverage. (This can also be
motivated by a probabilistic framework based on collections of
Poisson distributed random variables known as \emph{Poisson
processes,} see, for example, Klugman et al., 2008).

More generally, we wish to allow the mean to vary according to
information contained in other explanatory variables. For the
Poisson, it is customary to specify
\begin{equation*}
\mathrm{E~}y_i = \mu_i = \exp \left(
\mathbf{x}_i^{\prime}\boldsymbol \beta \right) .
\end{equation*}
Using the exponential function to map the systematic component
$\mathbf{x}_i^{\prime }\boldsymbol \beta$ into the mean ensures that
$\mathrm{E~}y_i$ will remain positive. Assuming the linearity of the
regression coefficients allows for easy interpretation.
Specifically, because
\begin{equation*}
\frac{\partial \mathrm{E~}y_i}{\partial x_{ij}} \times
\frac{1}{\mathrm{E~}y_i} =\beta_j,
\end{equation*}

\marginparjed{With a logarithmic link function, we may interpret
$\beta_j$\ to be the proportional change in the mean per unit change
in $x_j$.}

\index{link function!logarithmic}

\noindent we may interpret $\beta_j$\ to be the proportional change
in the mean per unit change in $x_{ij}$. The function that connects
the mean to the systematic component is known as the
\emph{logarithmic link function}, that is, $\ln
\mu_i=\mathbf{x}_i^{\prime }\boldsymbol \beta$.

To incorporate exposures, one can always specify one of the
explanatory variables to be $\ln E_i$ and restrict the corresponding
regression coefficient to be 1. This term is known as an
\emph{offset}. With this convention, the link function is
\begin{equation}\label{E12:logLink}
\ln \mu_i=\ln E_i+\mathbf{x}_i^{\prime }\boldsymbol \beta.
\end{equation}

\linejed\index{examples!California automobile
accidents}\index{offset}\index{link function}

\textbf{Example: California Automobile
Accidents.}\ecaptionjed{California Automobile Accidents} Weber
(1971) provided the first application of Poisson regression to
automobile accident frequencies in his study of California driving
records. In one model, Weber examined the number of automobile
accidents during 1963 of nearly 87,000 male drivers. His explanatory
variables consisted of:
\begin{itemize}
\item $x_1$ = the natural logarithm of the traffic
            density index of the county in which the driver resides,
\item $x_2 =5/(age-13)$
\item $x_3$ = the number of countable convictions incurred during
years 1961-62
\item $x_4$ = the number of accident involvements incurred during
years 1961-62
\item $x_5$ = the number of noncountable convictions incurred during
years 1961-62.
            \end{itemize}
Interestingly, in this early application, Weber achieved a
satisfactory fit representing the mean as a linear combination of
explanatory variables (E $y_i=\mathbf{x}_i^{\prime }\boldsymbol
\beta$), not the exponentiated version as in equation
(\ref{E12:logLink}) that is now commonly fit.


\linejed

\subsection{Estimation}\label{S12:Estimation}

Maximum likelihood is the usual estimation technique for Poisson
regression models. Using the logarithmic link function in equation
(\ref{E12:logLink}), the log-likelihood is given by
\begin{eqnarray*}
L(\boldsymbol \beta) &=&\sum_{i=1}^{n}\left( -\mu_i+y_i\ln \mu
_i-\ln y_i!\right) \\
&=&\sum_{i=1}^{n}\left( -E_i\exp \left( \mathbf{x}_i^{\prime
}\boldsymbol \beta \right) +y_i\left( \ln E_i+\mathbf{x}_i^{\prime
}\boldsymbol \beta \right) -\ln y_i!\right) .
\end{eqnarray*}
Setting the \emph{score function} equal to zero yields
\begin{equation}\label{E12:Score}
\left. \frac{\partial }{\partial \boldsymbol
\beta}\mathrm{L}(\boldsymbol \beta )\right\vert_{\mathbf{\beta
=b}}=\sum_{i=1}^{n}\left( y_i-E_i\exp \left( \mathbf{x}_i^{\prime
}\mathbf{b}\right) \right) \mathbf{x} _i=\sum_{i=1}^{n}\left(
y_i-\widehat{\mu }_i\right) \mathbf{x}_i= \mathbf{0},
\end{equation}
where $\widehat{\mu }_i = E_i\exp \left( \mathbf{x}_i^{\prime
}\mathbf{b} \right)$. Solving this equation (numerically) yields
$\mathbf{b}$, the maximum likelihood estimator of $\boldsymbol
\beta$. From equation (\ref{E12:Score}), we see that if a row of
$\mathbf{x}_i$ is constant (corresponding to a constant intercept
regression term), then the sum of residuals $y_i - \widehat{\mu}_i$\
is zero.

\marginparjed{In Poisson regression with an intercept, the sum, and
hence the average, of residuals is zero.}

\index{likelihood inference!information matrix}\index{likelihood
inference!score function}

Taking second derivatives yields the \emph{information matrix},
\begin{equation*}
\mathbf{I}(\boldsymbol \beta) = - \mathrm{E} \frac{\partial
^2}{\partial \boldsymbol \beta\partial \boldsymbol \beta^{\prime
}}\mathrm{L}(\boldsymbol \beta)=\sum_{i=1}^{n}E_i\exp \left(
\mathbf{x}_i^{\prime }\boldsymbol \beta\right)
\mathbf{x}_i\mathbf{x}_i^{\prime
}=\sum_{i=1}^{n}\mu_i\mathbf{x}_i\mathbf{x}_i^{\prime }.
\end{equation*}
Standard maximum likelihood estimation theory (Section 11.9.2) shows
that the asymptotic variance-covariance matrix of $\mathbf{b}$
is\index{matrix algebra!variance-covariance matrix}
\begin{equation*}
\widehat{\mathrm{Var~}\mathbf{b}}=\left(
\sum\limits_{i=1}^{n}\widehat{\mu }
_i\mathbf{x}_i\mathbf{x}_i^{\prime }\right)^{-1}.
\end{equation*}
The square root of the $j$th diagonal element of
$\widehat{\mathrm{Var~} \mathbf{b}}$ yields the standard error for
$b_j$, which we denote as $se(b_j)$.

\linejed\index{examples!medical malpractice insurance}

\textbf{Example: Medical Malpractice Insurance.}\ecaptionjed{Medical
Malpractice Insurance} Physicians make errors and may be sued by
parties harmed by these errors. Like many professionals, it is
common for physicians to carry insurance coverage that mitigates the
financial consequences of ``malpractice'' lawsuits.\index{actuarial
\& financial terms and concepts!malpractice
insurance}\index{actuarial \& financial terms and concepts!closed
claim}

Because insurers wish to accurately price this type of coverage, it
seems natural to ask what type of physicians are likely to submit
medical malpractice claims. Fournier and McInnes (2001) examined a
sample of $n=9,059$ Florida physicians using data from the Florida
Medical Professional Liability Insurance Claims File. The authors
examined closed claims in years 1985-1989 for physicians who were
licensed before 1981, thus omitting claims for newly licensed
physicians. Medical malpractice claims can take a long time to be
resolved (``settled''); in their study, Fournier and McInnes found
that 2 percent of claims were still not settled after 5 years of the
malpractice event. Thus, they chose an early period (1985-1989) to
allow the experience to mature. The authors also ignored minor
claims by only considering claims that exceeded \$100.

Table \ref{T12:MedMalPoisson} provides fitted Poisson regression
coefficients along with standard errors that appear in Fournier and
McInnes (2001). The table shows that physicians' practice area,
region, practice size and physician personal characteristics
(experience and gender) to be important determinants of the number
of medical malpractice suits. For example, we may interpret the
coefficient associated with gender to say that males are expected to
have $\exp (0.432)= 1.540$ times as many claims as females.

\begin{table}[h]
\scalefont{0.8}  \begin{center} \caption{\label{T12:MedMalPoisson}
Regression Coefficients of Medical Malpractice Poisson Regression
Model}
\begin{tabular}{lcc|lcc}
\hline &  & Standard  &
 & & Standard \\
Explanatory Variables & Coefficient & Error & Explanatory
Variables & Coefficient & Error \\
\hline Intercept & -1.634 & 0.254 & MSA: Miami Dade-Broward & 0.377
& 0.094
\\

Log Years Licensed & -0.392 & 0.054 & MSA: Other & 0.012 &
0.084 \\
\cline{4-6}
Female & -0.432 & 0.082 & \multicolumn{3}{c}{Speciality} \\

Patient Volume & 0.643 & 0.045 & Anesthesiology & 0.944 & 0.099 \\

(Patient Volume$)^2$& -0.066 & 0.008 & Emergency Medicine & 0.583
& 0.105 \\

Per Capita Education & -0.015 & 0.006 & Internal Medicine & 0.428 &
0.066 \\

Per Capita Income & 0.047 & 0.011 & Obstetrics-Gynecology & 1.226 &
0.070 \\
\cline{1-3} \multicolumn{3}{c}{Regional Variables} &
Otorhinolaryngology & 1.063 &
0.109 \\

Second Circuit & 0.066 & 0.072 & Pediatrics & 0.385 & 0.089 \\

Third Circuit & 0.103 & 0.088 & Radiology & 0.478 & 0.099 \\

Fourth Circuit & 0.214 & 0.098 & Surgery & 1.410 & 0.061 \\

Fifth Circuit & 0.287 & 0.069 & Other Specialties & 0.011 & 0.076
\\

 \hline
\end{tabular}

\end{center}  \scalefont{1.25}
\linetjed
\end{table}


\subsection{Additional
Inference}\label{S12:PoissonInference}\index{residual!Pearson}

In Poisson regression models, we anticipate \emph{heteroscedastic}
dependent variables because of the relation $\mathrm{Var~}y_i=\mu
_i$. This characteristic means that ordinary residuals
$y_i-\widehat{\mu }_i$\ are of less use, so that it is more common
to examine \emph{Pearson residuals}, defined as
\begin{equation*}
r_i=\frac{y_i-\widehat{\mu }_i}{\sqrt{\widehat{\mu }_i}}.
\end{equation*}
By construction, Pearson residuals are approximately homoscedastic.
Plots of Pearson residuals can be used to identify unusual
observations or to detect whether additional variables of interest
can be used to improve the model specification.

\index{goodness of fit statistics!Pearson chi-square}

Pearson residuals can also be used to calculate a Pearson goodness of fit
statistic,
\begin{equation}\label{E12:Pearson2}
\sum\limits_{i=1}^{n}r_i^2=\sum\limits_{i=1}^{n}\frac{\left( y_i-%
\widehat{\mu }_i\right)^2}{\widehat{\mu }_i}.
\end{equation}
This statistic is an overall measure of how well the model fits the
data. If the model is specified correctly, then this statistic
should be approximately $n-(k+1)$. In general, Pearson goodness of
fit statistics take the form $\sum \left( O-E\right)^2/E$, where $O$
is some observed quantity and $E$ is the corresponding estimated
(expected) value based on a model. The statistic in equation
(\ref{E12:Pearson2}) is computed at the observation level whereas
the statistic in equation (\ref{E12:Pearson}) was computed
summarizing information over cells.

In linear regression, the coefficient of determination $R^2$ is a
widely accepted goodness of fit measure. In nonlinear regression
such as for binary and count dependent variables, this is not true.
Information statistics, such as \emph{Akaike's Information
Criterion,}
\begin{equation*}
AIC=-2 L(\mathbf{b}) +2(k+1),
\end{equation*}
represents a type of statistic useful for goodness of fit that is
broadly defined over a large range of models. Models with smaller
values of \textit{AIC} fit better, and are preferred.

\index{goodness of fit statistics!Akaike's information criterion,
$AIC$}\index{hypothesis test!test statistics!likelihood ratio}

As noted in Section \ref{S12:Estimation}, \textit{t}-statistics are
regularly used for testing the significance of individual regression
coefficients. For testing collections of regression coefficients, it
is customary to use the \emph{likelihood ratio test}. The likelihood
ratio test is a well-known procedure for testing the null hypothesis
$H_0:\mathrm{h}(\boldsymbol \beta) = \mathbf{d}$, where $\mathbf{d}$
is a known vector of dimension $r\times 1$ and
$\mathrm{h}(\mathbf{.})$ is known and differentiable function. This
approach uses $\mathbf{b}$ and $\mathbf{b}_{\mathrm{Reduced}}$,
where $\mathbf{b}_{\mathrm{Reduced}}$ is the value of $\boldsymbol
\beta$ that maximizes $L(\boldsymbol \beta)$ under the restriction
that $\mathrm{h}(\boldsymbol \beta)=\mathbf{d}$. One computes the
test statistic
\begin{equation}\label{E12:LRT}
LRT = 2 \left( L(\mathbf{b}) - L(\mathbf{b}_{\mathrm{Reduced}})
\right) .
\end{equation}
Under the null hypothesis $H_0$, the test statistic $LRT$ has an
asymptotic chi-square distribution with $r$ degrees of freedom.
Thus, large values of $LRT$ suggest that the null hypothesis is not
valid.

\section{Application: Singapore Automobile Insurance}{\label{S12:SingaporeData}
\ecaptionjed{Singapore Automobile Insurance}

Frees and Valdez (2008) investigate hierarchical models of Singapore
driving experience. Here we examine in detail a subset of their
data, focussing on 1993 counts of automobile accidents. The purpose
of the analysis is to understand the impact of vehicle and driver
characteristics on accident experience. These relationships provide
a foundation for an actuary working in \emph{ratemaking}, that is,
setting the price of insurance coverages.

\index{actuarial \& financial terms and
concepts!ratemaking}\index{actuarial \& financial terms and
concepts!general insurers}

The data are from the General Insurance Association of Singapore, an
organization consisting of general (property and casualty) insurers
in Singapore (see the organization's website: www.gia.org.sg). From
this database, several characteristics were available to explain
automobile accident frequency. These characteristics include vehicle
variables, such as type and age, as well as person level variables,
such as age, gender and prior driving experience. Table
\ref{T12:Table122} summarizes these characteristics.

\scalefont{0.9}  \begin{center}  \begin{table}[h]
\caption{\label{T12:Table122} Description of Covariates}
\begin{tabular}{lp{4in}}
\hline \textbf{Covariate} & \multicolumn{1}{c}{\textbf{Description}}
\\ \hline Vehicle Type & \multicolumn{1}{|p{4in}}{The type of
vehicle being insured,
either automobile (A) or other (O).} \\
Vehicle Age & \multicolumn{1}{|p{4in}}{The age of the vehicle, in years,
grouped into six categories.} \\
Gender & \multicolumn{1}{|p{4in}}{The policyholder's gender, either male or
female} \\
Age & \multicolumn{1}{|p{4in}}{The age of the policyholder, in years,
grouped into seven categories.} \\
NCD & \multicolumn{1}{|p{4in}}{No Claims Discount. This is based on the
previous accident record of the policyholder.} \\
& \multicolumn{1}{|p{4in}}{The higher the discount, the better is the prior
accident record.} \\ \hline
\end{tabular}
\end{table}  \end{center}  \scalefont{1.1111}


Table \ref{T12:SingIntroStats} shows the effects of vehicle
characteristics on claim count. The ``Automobile'' category has
lower overall claims experience. The \textquotedblleft
Other\textquotedblright\ category consists primarily of (commercial)
goods vehicles, as well as weekend and hire cars. The vehicle age
shows nonlinear effects of the age of the vehicle. Here, we see low
claims for new cars with initially increasing accident frequency
over time. However, for vehicles in operation for long periods of
time, the accident frequencies are relatively low. There are also
some important interaction effects between vehicle type and age not
shown here. Nonetheless, Table \ref{T12:SingIntroStats} clearly
suggests the importance of these two variables on claim frequencies.

  \begin{center}  \begin{table}[h]
\caption{\label{T12:SingIntroStats}  Effect of Vehicle
Characteristics on Claims} \scalefont{0.9}
\begin{tabular}{crrrr|r}
\hline & Count=0 & Count=1 & Count=2 & Count=3 & Totals \\ \hline
\multicolumn{6}{l}{Vehicle Type} \\
Other & 3,441 & 184 & 13 & 3 & 3,641 \\
& (94.5) & (95.1) & (0.4) & (0.1) & (48.7) \\
Automobile & 3,555 & 271 & 15 & 1 & 3,842 \\
& (92.5) & (7.1) & (0.4) & (0.0) & (51.3) \\ \hline
\multicolumn{6}{l}{Vehicle Age (in years)} \\
0-2 & 4,069 & 313 & 20 & 4 & 4,406 \\
& (92.4) & (7.1) & (0.5) & (0.1) & (50.8) \\
3 to 5 & 708 & 59 & 4 &  & 771 \\
& (91.8) & (7.7) & (0.5) &  & (10.3) \\
6 to 10 & 872 & 49 & 3 &  & 924 \\
& (94.4) & (5.3) & (0.3) &  & (12.3) \\
11 to 15 & 1,133 & 30 & 1 &  & 1,164 \\
& (97.3) & (2.6)& (0.1) &  & (15.6) \\
16 and older & 214 & 4 &  &  & 218 \\
& (98.2) & (1.8)&  &  & (2.9) \\ \hline Totals & 6,996 & 455 & 28 &
4 & 7,483 \\ \hline \multicolumn{6}{l}{{\emph{Note:}} Number in
parens are
percentages.} \\
\end{tabular}\scalefont{1.1111}
\end{table}  \end{center}

Table \ref{T12:SingClaimsStats} shows the effects of person level
characteristics, gender, age and no claims discount, on the
frequency distribution. Person level characteristics were largely
unavailable for commercial use vehicles and so Table
\ref{T12:SingClaimsStats} present summary statistics for only those
observations having automobile coverage with the requisite gender
and age information. When we restricted consideration to (private
use) automobiles, relatively few policies did not contain gender and
age information.

Table \ref{T12:SingClaimsStats} suggests that driving experience was
roughly similar between males and females. This company insured very
few young drivers, so the young male driver category that typically
has extremely high accident rates in most automobiles studies is
less important for these data. Nonetheless, Table
\ref{T12:SingClaimsStats} suggests strong age effects, with older
drivers having better driver experience. Table
\ref{T12:SingClaimsStats} also demonstrates the importance of the no
claims discounts (NCD). As anticipated, drivers with better previous
driving records who enjoy a higher NCD have fewer accidents.

\begin{table}[h]\begin{center}
\caption{\label{T12:SingClaimsStats} Effect of Personal
Characteristics on Claims.\newline  Based on Sample with Auto=1.}
\scalefont{0.9}
\begin{tabular}{crr|r}
\hline
& \multicolumn{2}{c|}{Count=0} & Total \\
& Number & Percent &  \\ \hline
\multicolumn{4}{l}{Gender} \\
Female & \multicolumn{1}{|r}{654} & 93.4 & 700 \\
Male & \multicolumn{1}{|r}{2,901} & 92.3 & 3,142 \\ \hline
\multicolumn{4}{l}{Age Category} \\
22-25 & \multicolumn{1}{|r}{131} & 92.9 & 141 \\
26-35 & \multicolumn{1}{|r}{1,354} & 91.7 & 1,476 \\
36-45 & \multicolumn{1}{|r}{1,412} & 93.2 & 1,515 \\
46-55 & \multicolumn{1}{|r}{503} & 93.8 & 536 \\
56-65 & \multicolumn{1}{|r}{140} & 89.2 & 157 \\
\ \ \ \ \ \ \ \ \ \ \ 66 and over \ \ \ \ \ \ \ \  & \multicolumn{1}{|r}{15}
& 88.2 & 17 \\ \hline
\multicolumn{4}{l}{No Claims Discount} \\
0 & \multicolumn{1}{|r}{889} & 89.6 & 992 \\
10 & \multicolumn{1}{|r}{433} & 91.2 & 475 \\
20 & \multicolumn{1}{|r}{361} & 92.8 & 389 \\
30 & \multicolumn{1}{|r}{344} & 93.5 & 368 \\
40 & \multicolumn{1}{|r}{291} & 94.8 & 307 \\
50 & \multicolumn{1}{|r}{1,237} & 94.4 & 1,311 \\ \hline
Total & \multicolumn{1}{|r}{3,555} & 92.5 & 3,842 \\ \hline
\end{tabular}  \scalefont{1.1111} \end{center}
\end{table}

As part of the examination process, we investigated interaction
terms among the covariates and nonlinear specifications. However,
Table \ref{T12:SingPoissonEst} summarizes a simpler fitted Poisson
model with only additive effects. Table \ref{T12:SingPoissonEst}
shows that both vehicle age and no claims discount are important
categories in that the \textit{t}-ratios for many of the
coefficients are statistically significant. The overall
log-likelihood for this model is $L( \mathbf{b}) =-1,776.730$.

Omitted reference levels are given in the footnote of Table
\ref{T12:SingPoissonEst} to help interpret the parameters. For
example, for $NCD=0$, we expect that a poor driver with $NCD=0$ will
have $\exp (0.729)=2.07$ times as many accidents as a comparable
excellent driver with $NCD=50$. In the same vein, we expect that a
poor driver with $NCD=0$ will have $\exp (0.729-0.293)=1.55$ times
as many accidents as a comparable average driver with $NCD=30$.

 \begin{table}[h]\begin{center}
\caption{\label{T12:SingPoissonEst} Parameter Estimates from a
Fitted Poisson Model} \scalefont{0.9}
\begin{tabular}{rrr|rrr}
\hline
& Parameter &  &  & Parameter &  \\
Variable & Estimate & \textit{t}-ratio & Variable & Estimate & \textit{t}%
-ratio \\ \hline
\multicolumn{1}{c}{} &  &  & \multicolumn{3}{|c}{(Auto=1)$\times $No Claims
Discount*} \\
\multicolumn{1}{c}{Intercept} & -3.306 & -6.602 & \multicolumn{1}{|c}{0} &
0.729 & 4.704 \\
\multicolumn{1}{c}{Auto} & -0.667 & -1.869 & \multicolumn{1}{|c}{10} & 0.528
& 2.732 \\
\multicolumn{1}{c}{Female} & -0.173 & -1.115 & \multicolumn{1}{|c}{20} &
0.293 & 1.326 \\
\multicolumn{1}{c}{} &  &  & \multicolumn{1}{|c}{30} & 0.260 & 1.152 \\
\cline{1-3}
\multicolumn{3}{c|}{(Auto=1)$\times $Age Category*} & \multicolumn{1}{|c}{40}
& -0.095 & -0.342 \\ \cline{4-6}
\multicolumn{1}{c}{22-25} & 0.747 & 0.961 & \multicolumn{3}{|c}{Vehicle Age
(in years)*} \\
\multicolumn{1}{c}{26-35} & 0.489 & 1.251 & \multicolumn{1}{|c}{0-2} & 1.674
& 3.276 \\
\multicolumn{1}{c}{36-45} & -0.057 & -0.161 & \multicolumn{1}{|c}{3-5} &
1.504 & 2.917 \\
\multicolumn{1}{c}{46-55} & 0.124 & 0.385 & \multicolumn{1}{|c}{6-10} & 1.081
& 2.084 \\
\multicolumn{1}{c}{56-65} & 0.165 & 0.523 &
\multicolumn{1}{|c}{11-15} & 0.362 & 0.682 \\ \hline
\multicolumn{6}{l}{\small *The omitted reference levels are: \ ``66
and over'' for Age Category, ``50'' for} \\
\multicolumn{6}{c}{\small No Claims Discount and ``16 and over'' for
Vehicle Age.}
\end{tabular} \end{center} \scalefont{1.1111}
\end{table}

For a more parsimonious model, one might consider removing the
automobile, gender and age variables. Removing these seven variables
results in a model with a log-likelihood of $L \left(
\mathbf{b}_{\mathrm{Reduced}}\right) =-1,779.420$. To understand
whether this is a significant reduction, we can compute a likelihood
ratio statistic (equation \ref{E12:LRT}),
\begin{equation*}
LRT=2\times \left( -1,776.730 - (-1,779.420) \right) =5.379.
\end{equation*}
Comparing this to a chi-square distribution with $df=7$ degrees of
freedom, the statistic $p$-value $=\Pr \left( \chi
_{7}^2>5.379\right) =0.618$ indicates that these variables are not
statistically significant. Nonetheless, for purposes of further
model development, we retained automobile, gender and age as it is
customary to include these variables in ratemaking models.
\index{symbols!$\chi_k^2$, chi-square random variable with $k$
degrees of freedom}

As described in Section \ref{S12:PoissonInference}, there are
several ways of assessing a model's overall goodness of fit. Table
\ref{T12:SingGoodFit} compares several fitted models, providing
fitted values for each response level and summarizing the overall
fit with Pearson chi-square goodness of fit statistics. The left
portion of the table repeats the baseline information that appeared
in Table \ref{T12:Table121}, for convenience. \ To begin, first note
that even without covariates, the inclusion of the offset,
exposures, dramatically improves the fit of the model. This is
intuitively appealing; as a driver has more insurance coverage
during a year, he or she is more likely to be in an accident covered
under the insurance contract. Table \ref{T12:SingGoodFit} also shows
the improvement in the overall fit when including the fitted model
summarized in Table \ref{T12:SingPoissonEst}. When compared to a
chi-square distribution, the statistic $p$-value $=\Pr \left( \chi
_{4}^2>8.77\right) =0.067$ suggests agreement between the data and
the fitted value. However, this model specification can be improved
- the following section introduces a negative binomial model that
proves to be a yet better fit for this data set.

\begin{table}[h]\begin{center}
\caption{\label{T12:SingGoodFit} Comparison of Fitted Frequency
Models}\scalefont{0.9}
\begin{tabular}{crr|rrr}
\hline &  & Without & \multicolumn{3}{|c}{With Exposures} \\
\cline{4-6}
Count & Observed & Exposures/ & No & Poisson & Negative \\
&  & No Covariates & Covariates &  & Binomial \\ \hline
0 & 6,996 & 6,977.86 & 6,983.05 & 6,986.94 & 6,996.04 \\
1 & 455 & 487.70 & 477.67 & 470.30 & 453.40 \\
2 & 28 & 17.04 & 21.52 & 24.63 & 31.09 \\
3 & 4 & 0.40 & 0.73 & 1.09 & 2.28 \\
4 & 0 & 0.01 & 0.02 & 0.04 & 0.18 \\ \hline
\multicolumn{2}{c}{Pearson Goodness of Fit} & 41.98 & 17.62 & 8.77 &
1.79
\\ \hline
\end{tabular}\end{center}  \scalefont{1.1111}
\end{table}

\section{Overdispersion and Negative Binomial
Models}\label{S12:NBSection}\index{dispersion!equidispersion}\index{dispersion!overdispersion}
\index{dispersion!underdispersion}

Although simplicity is a virtue of the Poisson regression model, its form
can also be too restrictive. In particular, the requirement that the mean
equal the variance, known as \emph{equidispersion}, is not satisfied for
many datasets of interest. If the variance exceeds the mean, then the data
are said to be \emph{overdispersed}. A less common case occurs when the
variance is less than the mean, known as \emph{underdispersion}.

\subsubsection*{Adjusting Standard Errors for Data not Equidispersed}

To mitigate this concern, a common specification is to assume that
\begin{equation}\label{E12:Pscale}
\mathrm{Var~}y_i=\phi \mu_i,
\end{equation}
where $\phi >0$\ is a parameter to accommodate the potential over-
or under-dispersion. As suggested by equation (\ref{E12:Score}),
consistent estimation of $\boldsymbol \beta$ requires only that the
mean function be specified correctly, not that the equidispersion or
Poisson distribution assumptions hold. This feature also holds for
linear regression. Because of this, the estimator $\mathbf{b}$ is
sometimes referred to as a \emph{quasi-likelihood estimator}. With
this estimator, we may compute estimated means $\widehat{\mu}_i$\
and then estimate $\phi $\ as\
\begin{equation}\label{E12:QuasiEstimator}
\widehat{\phi }=\frac{1}{n-(k+1)}\sum\limits_{i=1}^{n}\frac{\left( y_i-%
\widehat{\mu }_i\right)^2}{\widehat{\mu }_i}.
\end{equation}
Standard errors are then based on
\begin{equation*}
\widehat{\mathrm{Var~}\mathbf{b}}=\left( \widehat{\phi }\sum%
\limits_{i=1}^{n}\widehat{\mu }_i\mathbf{x}_i\mathbf{x}_i^{\prime
}\right)^{-1}.
\end{equation*}\index{likelihood inference!quasi-likelihood estimator}
\index{likelihood inference!robust standard error}

A drawback of equation (\ref{E12:Pscale}) is that one assumes the
variance of each observation is a constant multiple of its mean. For
datasets where this assumption is in doubt, it is common to use a
\emph{robust standard error}, computed as the square root of the
diagonal element of
\begin{equation*}
\mathrm{Var~}\mathbf{b}=\left( \sum\limits_{i=1}^{n}\mu_i\mathbf{x}_i%
\mathbf{x}_i^{\prime }\right)^{-1}\left( \sum\limits_{i=1}^{n}\left(
y_i-\mu_i\right)^2\mathbf{x}_i\mathbf{x}_i^{\prime }\right) \left(
\sum\limits_{i=1}^{n}\mu_i\mathbf{x}_i\mathbf{x}_i^{\prime
}\right)^{-1},
\end{equation*}
evaluated at $\widehat{\mu }_i.$ Here, the idea is that $\left(
y_i-\mu_i\right)^2$\ is an unbiased estimator of Var $y_i$,
regardless of the form. Although $\left( y_i-\mu_i\right)^2$\ is a
poor estimator of Var $y_i$ for each observation $i$, the weighted
sum $ \sum\nolimits_i\left( y_i-\mu_i\right)^2\mathbf{x}_i\mathbf{x}
_i^{\prime }$\ is a reliable estimator of $\sum\nolimits_i\left(
\mathrm{Var~}y_i\right) \mathbf{x}_i\mathbf{x}_i^{\prime }$.

For the quasi-likelihood estimator, the estimation strategy assumes only a
correct specification of the mean and uses a more robust specification of
the variance than implied by the Poisson distribution. The advantage and
disadvantage of this estimator is that it is not linked to a full
distribution. This assumption makes it difficult, for example, if the
interest is in estimating the probability of zero counts. An alternative
approach is to assume a more flexible parametric model that permits a wider
range of dispersion.

\subsubsection*{Negative Binomial}\index{distributions!negative binomial}

A widely used model for counts is the \emph{negative binomial}, with
probability mass function
\begin{equation}
\mathrm{Pr}(y=j)=\left(
\begin{array}{c}
j+r-1 \\
r-1
\end{array}
\right) p^{r}\left( 1-p\right)^j,
\end{equation}
where $r$ and $p$ are parameters of the model. To help interpret the
parameters of the model, straightforward calculations show that
$\mathrm{E~}y=r(1-p)/p$ and $\mathrm{Var~}y = r(1-p)/p^2.$

The negative binomial has several important advantages when compared
to the Poisson distribution. First, because there are two parameters
describing the negative binomial distribution, it has greater
flexibility for fitting data. Second, it can be shown that the
Poisson is a limiting case of the negative binomial (by allowing
$p\rightarrow 1$ and $r \rightarrow 0$ such that $rp \rightarrow
\lambda $). In this sense, the Poisson is nested within the negative
binomial distribution. Third, one can show that negative binomial
distribution arises from a mixture of the Poisson variables. For
example, think about the Singapore data set with each driver having
their own value of $\lambda $. Conditional on $\lambda $, assume
that the driver's accident distribution has a Poisson distribution
with parameter $\lambda $. Further assume that the distribution of
$\lambda $'s can be described as a gamma distribution. Then, it can
be shown that the overall accident counts have a negative binomial
distribution. See, for example, Klugman et al. (2008). Such
``mixture'' interpretations are helpful in explaining results to
consumers of actuarial analyses.

For regression modeling, the ``$p$'' parameter varies by subject
$i$. It is customary to reparameterize the model and use a log-link
function such that $\sigma =1/r$ and that $p_i$ related to the mean
through $\mu_i =r(1-p_i)/p_i = \exp (\mathbf{x}_i^{\prime}
\boldsymbol \beta)$. Because the negative binomial is a probability
frequency distribution, there is no difficulty in estimating
features of this distribution, such as the probability of zero
counts, after a regression fit. This is in contrast to the
quasi-likelihood estimation of a Poisson model with an ad hoc
specification of the variance summarized in equation
(\ref{E12:QuasiEstimator}).

\linejed

\textbf{Example: Singapore Automobile Data - Continued.} The
negative binomial distribution was fit to the Section
\ref{S12:SingaporeData} Singapore data using the set of covariates
summarized in Table \ref{T12:SingPoissonEst}. The resulting
log-likelihood was $\mathrm{L}_{NegBin}(\mathbf{b})=-1,774.494;$
this is
larger than the Poisson likelihood fit $\mathrm{L}_{Poisson}\left( \mathbf{b}%
\right) =-1,776.730$ because of an additional parameter. The usual
likelihood ratio test is not formally appropriate because the models
are only nested in a limiting sense. It is more useful to compare
the goodness of fit statistics given in Table \ref{T12:SingGoodFit}.
Here, we see that the negative binomial is a better fit than the
Poisson (with the same systematic components). A chi-square test of
whether the negative binomial with covariates is suitable yields
$p$-value\textrm{\ }$=\Pr \left( \chi_{4}^2>1.79\right) =0.774$,
suggesting strong agreement between the observed data and fitted
values. We interpret the findings of Table \ref{T12:SingGoodFit} to
mean that the negative binomial distribution well captures the
heterogeneity in the accident frequency distribution.

\linejed

\section{Other Count Models}

Actuaries are familiar with a host of frequency models; see, for
example, Klugman et al. (2008). In principle, each frequency model
could be used in a regression context by simply incorporating a
systematic component, $\mathbf{x}^{\prime}\boldsymbol \beta$, into
one or more model parameters. However, analysts have found that four
variations of the basic models perform well in fitting models to
data and provide an intuitive platform for interpreting model
results.

\subsection{Zero-Inflated Models}\index{regression model!count!zero-inflated}

For many datasets, a troublesome aspect is the ``excess'' number of
zeros, relative to a specified model. For example, this could occur
in automobile claims data because insureds are reluctant to report
claims, fearing that a reported claim will result in higher future
insurance premiums. Thus, we have a higher than anticipated number
of zeros due to the non-reporting of claims.

A zero-inflated model represents the claims number $y_i$ as a
mixture of a point mass at zero and another claims frequency
distribution, say $g_i(j)$ (which is typically Poisson or negative
binomial). (We might interpret the point mass as the tendency of
non-reporting.) The probability of getting the point mass would be
modeled by a binary count model such as, for example, the logit
model
\begin{equation*}
\pi_i=\frac{\exp \left( \mathbf{x}_i^{\prime}\boldsymbol \beta%
_{1}\right) }{1+\exp \left( \mathbf{x}_i^{\prime}\boldsymbol \beta%
_{1}\right) }.
\end{equation*}
As a consequence of the mixture assumption, the zero-inflated count
distribution can be written as
\begin{equation}\label{E12:ZICount}
\Pr \left( y_i=j\right) =\left\{
\begin{array}{ll}
\pi_i+(1-\pi_i)g_i(0) & j=0 \\
(1-\pi_i)g_i(j) & j=1,2,...
\end{array}
\right. .
\end{equation}
From equation (\ref{E12:ZICount}), we see that zeros could arise
from either the point mass or the other claims frequency
distribution.

To see the effects of a zero-inflated model, suppose that $g_i$
follows a Poisson distribution with mean $\mu_i$. Then, easy
calculations show that
\begin{equation*}
\mathrm{E~} y_i = (1 - \pi_i) \mu_i
\end{equation*}
and%
\begin{equation*}
\mathrm{Var~} y_i = \pi_i \mu_i+\pi_i\mu_i^2(1-\pi_i).
\end{equation*}%
Thus, for the zero-inflated Poisson, the variance always exceeds the
mean, thus accommodating overdispersion relative to the Poisson
model.

\linejed\index{examples!automobile insurance}

\textbf{Example: Automobile Insurance}\ecaptionjed{Automobile
Insurance}. Yip and Yau (2005) examine a portfolio of $n=2,812$
automobile policies available from SAS Institute, Inc. \ Explanatory
variables include age, gender, marital status, annual income, job
category and education level of the policyholder. For this dataset,
they found that several zero-inflated count models accommodated well
the presence of extra zeros.

\linejed

\subsection{Hurdle Models}\index{regression model!count!hurdle}

A ``hurdle model'' provides another mechanism to modify basic count
distributions in order to represent situations with an excess number
of zeros. Hurdle models can be motivated by sequential decision
making processes confronted by individuals. For example, in
healthcare choice, we can think about an individual's decision to
seek healthcare care as an initial process. Conditional on having
sought healthcare $\{y\geq 1\}$, the amount of healthcare is a
decision made by a healthcare provider (such as a physician or
hospital), thus representing a different process. One needs to pass
the first ``hurdle'' (the decision to seek healthcare) in order to
address the second (the amount of healthcare). An appeal of the
hurdle model is its connection to the ``principal-agent'' model
where the provider (agent) decides on the amount after initial
contact by the insured (principal) is made. As another example, in
property and casualty insurance, the decision process an insured
uses for reporting the initial claim may differ from that used for
reporting subsequent claims.

To represent hurdle models, let $\pi_i$\ represent the probability that $%
\{y_i=0\}$ used for the first decision and suppose that $g_i$
represents the count distribution that will be used for the second
decision. We define the probability mass function as
\begin{equation} \label{E12:Hurdle}
\Pr \left( y_i=j\right) =\left\{
\begin{array}{ll}
\pi_i & j=0 \\
k_ig_i(j) & j=1,2,...%
\end{array}%
\right. .
\end{equation}%
where $k_i=(1-\pi_i)/(1-g_i(0))$. As with zero-inflated models, a
logit model might be suitable for representing $\pi_i$.

To see the effects of a hurdle model, suppose that $g_i$ follows a
Poisson distribution with mean $\mu_i$. Then, easy calculations show
that
\begin{equation*}
\mathrm{E~} y_i =k_i \mu_i
\end{equation*}
and%
\begin{equation*}
\mathrm{Var~} y_i = k_i \mu_i + k_i \mu_i^2(1-k_i).
\end{equation*}
Because $k_i$\ may be larger or smaller than 1, this model allows
for both under- and overdispersion relative to the Poisson model.

The hurdle model is a special case of the ``two-part'' model
described in Chapter 16. There, we will see that for two-part
models, the amount of healthcare utilized may be a continuous as
well as a count variable. An appeal of two-part models is that
parameters for each hurdle/part can be analyzed separately.
Specifically, the log-likelihood for the i$th$ subject can be
written as
\begin{equation*}
\ln \left[ \Pr \left( y_i=j\right) \right] =\left[
\mathrm{I}(j=0)\ln \pi_i+\mathrm{I}(j\geq 1)\ln (1-\pi_i)\right]
+\mathrm{I}(j\geq 1)\ln \frac{g_i(j)}{(1-g_i(0))}.
\end{equation*}%
The terms in the square brackets on the right-hand side correspond
to the likelihood for a binary count model. The latter terms
correspond to a count model with zeros removed (known as a truncated
model). If the parameters for the two pieces are different
(``separable''), then the maximization may be done separately for
each part.

\subsection{Heterogeneity Models}\index{regression model!count!heterogeneity}

In a heterogeneity model, one allows one or more model parameters to
vary randomly. The motivation is that these random parameters
capture unobserved features of a subject. For example, suppose that
$\alpha_i$ represents a random parameter and that $y_i$\ given
$\alpha_i$ has conditional mean $ \exp \left(
\alpha_i+\mathbf{x}_i^{\prime}\boldsymbol \beta \right) .$ We
interpret $\alpha_i,$ called a \emph{heterogeneity component, }to
represent unobserved subject characteristics that contribute
linearly to the systematic component
$\mathbf{x}_i^{\prime}\boldsymbol \beta$.

To see the effects of the heterogeneity component on the count distribution,
basic calculations show that
\begin{equation*}
\mathrm{E~} y_i = \exp \left( \mathbf{x}_i^{\prime} \boldsymbol
\beta \right) =\mu_i
\end{equation*}
and%
\begin{equation*}
\mathrm{Var~} y_i = \mu_i + \mu_i^2 \mathrm{Var}\left(
e^{\alpha_i}\right) .
\end{equation*}
where we typically assume that $\mathrm{E}\left( e^{\alpha
_i}\right) =1$\ for parameter identification. Thus, heterogeneity
models readily accommodate overdispersion in datasets.

It is common to assume that the count distribution is Poisson
conditional on $\alpha_i$. There are several choices for the
distribution of $\alpha_i$, the two most common being the log-gamma
and the log-normal. For the former, one first assumes that $\exp
\left( \alpha_i\right) $\ has a gamma distribution, implying that
$\exp \left( \alpha_i + \mathbf{x}_i^{\prime} \boldsymbol
\beta\right) $\ also has a gamma distribution. Recall that we have
already noted in Section \ref{S12:NBSection} that using a gamma
mixing distribution for Poisson counts results in a negative
binomial distribution. Thus, this choice provides another motivation
for the popularity of the negative binomial as the choice of the
count distribution. For the latter, assuming that an observed
quantity such as $\exp \left( \alpha_i\right) $\ has a normal
distribution is quite common in applied data analysis. Although
there are no closed-form analytic expressions for the resulting
marginal count distribution, there are several software packages
that readily lend itself to ease computational difficulties.

The heterogeneity component is particularly useful in repeated
samples where it can be used to model clustering of observations.
Observations from different clusters tend to be dissimilar compared
to observations within a cluster, a feature known as
\textquotedblleft heterogeneity.\textquotedblright\ \ The similarity
of observations within a cluster can be captured by a common term
$\alpha_i$. Different heterogeneity terms from observations from
different clusters can capture the heterogeneity. For an
introduction to modeling from repeated sampling, see Chapter 10.

\linejed\index{examples!Spanish third party automobile liability
insurance}

\textbf{Example: Spanish Third Party Automobile Liability
Insurance.}\ecaptionjed{Spanish Third Party Automobile Liability
Insurance} Boucher et al. (2006) analyzed a portfolio of $n=548,830$
automobile contracts from a major insurance company operating in
Spain. Claims were for third party automobile liability, so that in
the event of an automobile accident, the amount that the insured is
liable for non-property damages to other parties is covered under
the insurance contract. For these data, the average claims frequency
was approximately 6.9\%. Explanatory variables include age, gender,
driving location, driving experience, engine size and policy type.
The paper considers a wide variety of zero-inflated, hurdle and
heterogeneity models, showing that each was a substantial
improvement over the basic Poisson model.

\linejed

\subsection{Latent Class Models}\index{regression model!count!latent class}

In most data sets, it is easy to think about classifications of subjects
that the analyst would like to make in order to promote homogeneity among
observations. Some examples include:

\begin{itemize}
\item ``healthy'' and ``ill'' people when examining healthcare expenditures,

\item automobile drivers who are likely to file a claim in the event of an
accident compared to those who are reluctant to do so and

\item physicians who are ``low'' risks
compared to ``high'' risks when examining medical malpractice
insurance coverage.
\end{itemize}

For many datasets of interests, such obvious classification
information is not available and are said to be unobserved, or
\emph{latent}. A ``latent class'' model still employs this
classification idea but treats it as an unknown discrete random
variable. Thus, like Sections 12.4.1-12.4.3, we use mixture models
to modify basic count distributions but now assume that the mixture
is a discrete random variable that we interpret to be the latent
class.

To be specific, assume that we have two classes, ``low-risk'' and
``high-risks,'' with probability $\pi_{L}$ that a subject belongs to
the low-risk class. Then, we can write the probability mass
function as%
\begin{equation}\label{E12:Latent}
\Pr \left( y_i=j\right) =\pi_{L}\Pr \left( y_i=j;L\right) +\left(
1-\pi_{L}\right) \Pr \left( y_i=j;H\right) ,
\end{equation}%
where $\Pr \left( y_i=j;L\right) $\ and $\Pr \left( y_i=j;H\right)
$\ are the probability mass functions for the low and high risks,
respectively.

This model is intuitively pleasing in that corresponds to an
analyst's perception of the behavior of the world. It is flexible in
the sense that the model readily accommodates under- and
over-dispersion, long-tails and bi-modal distributions. However,
this flexibility also leads to difficulty regarding computational
issues. There is a possibility of multiple local maxima when
estimating via maximum likelihood. Convergence can be slow compared
to other methods described in 12.4.1-12.4.3.

Nonetheless, latent class models have proven fruitful in
applications of interest to actuaries.

\linejed\index{examples!Rand health insurance experiment}

\textbf{Example: Rand Health Insurance Experiment.}\ecaptionjed{Rand
Health Insurance Experiment} Deb and Trivedi (2002) find strong
evidence that a latent class model performs well when compared to
the hurdle model. They examined counts of utilization of healthcare
expenditures for the Rand Health Insurance Experiment, a dataset
that has extensively analyzed in the health economics literature.
They interpreted $\Pr \left( y_i=j;L\right) $\ to be a distribution
of infrequent healthcare users and $\Pr \left( y_i=j;H\right) $\ to
be a distribution of \ frequent healthcare users. Each distribution
was based on a negative binomial distribution, with different
parameters for each class. They found statistically significant
differences for their four insurance variables, two coinsurance
variables, a variable indicating whether there was an individual
deductible and a variable describing the maximum limit reimbursed.
Because subjects were randomly assigned to insurance plans (very
unusual), the effects of insurance variables on healthcare
utilization are particularly interesting from a policy standpoint,
as are differences among low and high use subjects. For their data,
they estimated that approximately 20\% were in the high use class.

\linejed

\section{Further Reading and References}

The Poisson distribution was derived by Poisson (1837) as a limiting
case of the binomial distribution. Greenwood and Yule (1920) derived
the negative binomial distribution as a mixture of a Poisson with a
gamma distribution. Interestingly, one example of the 1920 paper was
to use the Poisson distribution as a model of accidents, with the
mean as a gamma random variable, reflecting the variation of workers
in a population. Greenwood and Yule referred to this as individuals
subject to ``repeated accidents'' that other authors have dubbed as
``accident-proneness.''

The first application of Poisson regression is due to Cochran (1940)
in the context of ANOVA modeling and to Jorgensen (1961) in the
context of multiple linear regression. As described in Section 12.2,
Weber (1971) gives the first application to automobile accidents.

This chapter focuses on insurance and risk management applications
of count models. For those interested in automobiles, there is a
related literature on studies of motor vehicle crash process, see
for example, Lord et al. (2005). For applications in other areas of
social science and additional model development, we refer to Cameron
and Trivedi (1998).

\bigskip

\textbf{References}


\scalefont{0.9}

\begin{multicols}{2}

Bortkiewicz, L. von (1898). \textit{Das Gesetz de Kleinen Zahlen}.
Leipzig, Teubner.

Boucher, Jean-Philippe, Michel Denuit and Montserratt Guill\'{e}n
(2006). Risk classification for claim counts: A comparative analysis
of various zero-inflated mixed Poisson and hurdle models. Working
paper.

Cameron, A. Colin and Pravin K. Trivedi. (1998) \textit{Regression Analysis
of Count Data}. Cambridge University Press, Cambridge.

Cochran, W. G. (1940). The analysis of variance when experimental
errors follow the Poisson or binomial law. \textit{Annals of
Mathematical Statistics} 11, 335-347.

Deb, Partha and Pravin K. Trivedi (2002). The structure of demand for health
care: latent class versus two-part models. \textit{Journal of Health
Economics} 21, 601-625.

Fournier, Gary M. and Melayne Morgan McInnes (2001). The case of
experience rating in medical malpractice insurance: An empirical
evaluation. \textit{The Journal of Risk and Insurance} 68, 255-276.

Frees, Edward W. and Emiliano Valdez (2008). Hierarchical insurance
claims modeling. \textit{Journal of the American Statistical
Association} 103, 1457-1469.

Greenwood, M. and G. U. Yule (1920). An inquiry into the nature of
frequency distributions representative of multiple happenings with
particular reference to the occurrence of multiple attacks of
disease or of repeated accidents. \textit{Journal of the Royal
Statistical Society} 83, 255-279.

Jones, Andrew M. (2000). Health econometrics. Chapter 6 of the\ \textit{%
Handbook of Health Economics, Volume 1}. Edited by Antonio.J. Culyer, and
Joseph.P. Newhouse, Elsevier, Amersterdam. 265-344.

Jorgensen, Dale W. (1961). Multiple regression analysis of a Poisson
process.  \textit{Journal of the American Statistical Association}
56, 235-245.

Lord, Dominique, Simon P. Washington and John N. Ivan (2005).
Poisson, Poisson-gamma and zero-inflated regression models of motor
vehicle crashes: Balancing statistical theory and fit.
\textit{Accident Analysis and Prevention} 37, 35-46.

Klugman, Stuart A, Harry H. Panjer and Gordon E. Willmot (2008).
\emph{Loss Models: From Data to Decisions}. John Wiley \& Sons,
Hoboken, New Jersey.

Purcaru, Oana and Michel Denuit (2003). Dependence in dynamic claim
frequency credibility models. \textit{ASTIN Bulletin} 33(1), 23-40.

Weber, Donald C. (1971). Accident rate potential: An application of
multiple regression analysis of a Poisson process. \textit{Journal
of the American Statistical Association} 66, 285-288.

Yip, Karen C. H. and Kelvin K.W. Yau (2005). On modeling claim frequency
data in general insurance with extra zeros. \textit{Insurance: Mathematics
and Economics }36(2) 153-163.


\end{multicols}


\scalefont{1.1111}

\section{Exercises}

\begin{exercises}

\scalefont{0.90}

\item  Show that the log-likelihood in equation (\ref{E12:BasicLogLike})
has a maximum at $ \widehat{\mu }=\overline{y}$.

\item  For the data in Table \ref{T12:Table121}, confirm that the
Pearson statistic in equation (\ref{E12:Pearson}) is 41.98.

\item \textbf{Poisson Residuals}. Consider a Poisson regression. Let $e_i
= y_i - \widehat{\mu}_i$ denote the $i$th ordinary residual. Assume
that an intercept is used in the model so that one of the
explanatory variables $x$ is a constant equal to one.

a. Show that the average ordinary residual is 0.

b. Show that the correlation between the ordinary residuals and each
explanatory variable is zero.



\item  \textbf{Negative Binomial Distribution}.

a. Assume that $y_1, \ldots, y_n$ are i.i.d. with a negative
binomial distribution with parameters $r$ and $p$. Determine the
maximum likelihood estimators.

b. Use the sampling mechanism in part (a) but with parameters
$\sigma =1/r$ and $\mu$ where $\mu =r(1-p)/p.$ Determine the maximum
likelihood estimators of $\sigma$ and $\mu.$

c. Assume that $y_1, \ldots, y_n$ are independent with $y_i$ having
a negative binomial distribution with parameters $r$ and $p_i$,
where $\sigma =1/r$ and $p_i$ satisfies $r(1-p_i)/p_i=\exp
(\mathbf{x}_i^{\prime }\boldsymbol \beta) (= \mu_i).$ Determine the
score function in terms of $\sigma$ and $\boldsymbol \beta$.

\empexjed{HealthExpend}\index{datasets!MEPS health expenditures}

\item   \textbf{Medical Expenditures Data.} This exercise considers data
from the Medical Expenditure Panel Survey (MEPS) described in
Exercise 1.\ref{Ex:MedExpend} and Section 11.4. Our dependent
variable consists of the number of outpatient (COUNTOP) visits. For
MEPS, outpatient events include hospital outpatient department
visits, office-based provider visits and emergency room visits
excluding dental services. (Dental services, compared to other types
of health care services, are more predictable and occur in a more
regular basis.) Hospital stays with the same date of admission and
discharge, known as ``zero-night stays,'' were also included in
outpatient counts and expenditures. (Payments associated with
emergency room visits that immediately preceded an inpatient stay
were included in the inpatient expenditures. Prescribed medicines
that can be linked to hospital admissions were included in inpatient
expenditures, not in outpatient utilization.)

Consider the explanatory variables described in Section 11.4.

a. Provide a table of counts, a histogram and summary statistics of
COUNTOP. Note the shape of the distribution and the relationship
between the sample mean and sample variance.

b. Create tables of means of COUNTOP by level of GENDER, ethnicity,
region, education, self-rated physical health, self-rated mental
health, activity limitation, income and insurance. Do these tables
suggest that these explanatory variables have an impact on COUNTOP?

c. As a baseline, estimate a Poisson model without any explanatory
variables and calculate a Pearson's chi-square statistic for
goodness of fit (at the individual level).

d. Estimate a Poisson model using the explanatory variables in part
(b).

d(i). Comment briefly on the statistical significance of each
variable.

d(ii). Provide an interpretation for the GENDER coefficient.

d(iii). Calculate a (individual-level) Pearson's chi-square
statistic for goodness of fit. Compare this to the one in part (b).
Based on this statistic and the statistical significance of
coefficients discussed in part d(i), which model do you prefer?

d(iv). Re-estimate the model using the quasi-likelihood estimator of
the dispersion parameter. How have your comments in part d(i)
changed?

e. Estimate a negative binomial model using the explanatory
variables in part (d).

e(i). Comment briefly on the statistical significance of each
variable.

e(ii). Calculate a (individual-level) Pearson's chi-square statistic
for goodness of fit. Compare this to the ones in parts (b) and (d).
Which model do you prefer? Also cite the $AIC$ statistic in your
comparison.

e(iii). Re-estimate the model, dropping the factor income. Use the
likelihood ratio test to say whether income is a statistically
significant factor.

f. As a robustness check, estimate a logistic regression model using
the explanatory variables in part (d). Do the signs and significance
of the coefficients of this model fit give the same interpretation
as with the negative binomial model in part (e)?


\item \textbf{Two Population Poissons.} We can express the two
population problem in a regression context using one explanatory
variable. Specifically, suppose that $x_i$ only takes on the values
0 and 1. Out of the $n$ observations, $n_0$ take on the value $x=0$.
These $n_0 $ observations have an average $y$ value of
$\overline{y}_0$. The remaining $n_1 =n-n_0$ observations have value
$x=1$ and an average $y$ value of $\overline{y}_1$.

Use the Poisson model with the logarithmic link function and
systematic component $\mathbf{x}_i^{\prime} \boldsymbol \beta =
\beta_0 +\beta_1 x_i$.

i. Determine the maximum likelihood estimators of $\beta_0$ and
$\beta_1$, respectively.

ii. Suppose that $n_0 = 10$, $n_1= 90$, $\overline{y}_0 = 0.20$ and
$\overline{y}_1= 0.05$. Using your results in part a(i), compute the
maximum likelihood estimators of $\beta_0$ and $\beta_1$,
respectively.

iii. Determine the information matrix.


\scalefont{1.1111}

\end{exercises}

% cup2egui.tex (LaTeX 2e version)
% v1.01 --- released 9th July 1997
% v1.0  --- released 9th May 1997
%           based on cupguide.tex v1.2 (for LaTeX2.09, 27.4.95)

\NeedsTeXFormat{LaTeX2e}[1996/06/01]

\documentclass[cup6a]{cupbook}

\title[ \LaTeXe\ Style Guide for Authors]
      {CUP Standard Designs}
\author{Cambridge \TeX-to-type}
\date{9th July 1997}

\begin{document}

\pagenumbering{roman}
\maketitle
\tableofcontents
\cleardoublepage
\pagenumbering{arabic}

\chapter{\LaTeXe\ style guide for authors}

This guide is for authors who are preparing a book for CUP using the
\LaTeX\ document preparation system and the CUPBOOK class file.

\section{Introduction}

The \LaTeX\ document preparation system is a special version of the
\TeX\ typesetting program.
 \LaTeX\ adds to \TeX\ a collection of commands which simplify
typesetting by allowing the author to concentrate on the
logical structure of the document rather than its visual layout.

\LaTeX\ provides a consistent and comprehensive document preparation
interface.
 There are simple-to-use commands for generating a table of contents,
lists of figures and/or tables, and indexes.
 \LaTeX\ can automatically number list entries, equations, figures,
tables, and footnotes, as well as parts, chapters, sections and
subsections.
 Using this numbering system, bibliographic citations, page references
and cross references to any other numbered entity (e.g. chapter,
section, equation, figure, list entry) are quite straightforward.

\LaTeX\ is a powerful tool for managing long and complex documents.
In particular, partial processing enables long documents to be
produced chapter by chapter without losing sequential information.
The use of document classes allows a simple change of style (or style
option) to transform the appearance of your document.

\section{The CUPBOOK document class}

The CUP standard designs have been implemented as a \LaTeXe\ class file.
The CUPBOOK class file is based on the BOOK class as discussed in the
\LaTeX\ manual.
 Commands which differ from the standard \LaTeX\ interface, or which
are provided in addition to the standard interface, are explained in
this guide.
 This guide is \emph{not} a substitute for the \LaTeX\ manual itself.

\subsection{The BOOK class}

The CUPBOOK class file preserves the standard \LaTeX\ interface such
that any document which can be  produced using the standard \LaTeX\
BOOK class can also be produced with the CUPBOOK class.
 However, the measure (i.e. width of text) is different from that for
BOOK, therefore line-breaks will change and long equations may
need re-setting.

\subsection{Using the CUPBOOK class}

First, copy the file \verb"cupbook.cls" into the correct subdirectory
on your system.
The CUPBOOK document class is implemented as a complete docu\-ment
class, \emph{not} a document class option.
In order to use the CUPBOOK class, replace \verb"book" by
\verb"cupbook" in the \verb"\documentclass" command at the beginning
of your document:
 \begin{verbatim}
  \documentclass{book}
\end{verbatim}
 is replaced by
 \begin{verbatim}
  \documentclass{cupbook}
\end{verbatim}
 In general, the following standard document class options should \emph{not}
 be used:
 \begin{itemize}\listsize
  \item \texttt{10pt}, \texttt{11pt}, \texttt{12pt} -- unavailable.
  \item \texttt{oneside} (no associated file) -- \texttt{twoside} is the default.
  \item \texttt{fleqn}, \texttt{leqno}, \texttt{titlepage}, \texttt{twocolumn} --
        should not be used -- \texttt{fleqn} is incorporated into the indented
        design options.
  \item \texttt{proc}, \texttt{ifthen} -- these packages can be used if necessary.
 \end{itemize}

\subsection{Class options}

The following class options correspond to the standard designs which
are available with the CUPBOOK class:
 \begin{itemize}\listsize
  \item \texttt{cup5a} -- a single-author, indented design with a large format.
  \item \texttt{cup5b} -- a multi-author, indented design with a large format.
  \item \texttt{cup6a} -- a single-author, centred design with a large format.
  \item \texttt{cup6b} -- a multi-author, centred design with a large format.
  \item \texttt{cup7a} -- a single-author, centred design with a small
        format (default).
  \item \texttt{cup7b} -- a multi-author, centred design with a small format.
  \item \texttt{cup8a} -- a single-author, indented design with a small format.
  \item \texttt{cup8b} -- a multi-author, indented design with a small format.
  \item \texttt{cup9a} -- a single-author, flush-left design with a small format.
  \item \texttt{cup9b} -- a multi-author, flush-left design with a small format.
  \item \texttt{caps}  -- Cambridge Astrophysics Series
 \end{itemize}
If no option is specified, the CUP7A class option is produced.

\subsection{Landscape pages}
\label{land}

An additional class option is available with the CUPBOOK class:
 \begin{itemize}\listsize
  \item \texttt{landscape} -- for producing wide figures and tables which
 need to be included in landscape format (i.e. sideways) rather
 than portrait (i.e. upright).
 \end{itemize}
 If a table or illustration is too wide to fit the standard measure, it
must be turned through 90 degrees anti-clockwise, with its caption, on
a page by itself with no running head or page number.
 Landscape illustrations and/or tables cannot be produced directly
using the CUPBOOK class file because \TeX\ itself cannot turn the
page, and not all device drivers provide such a facility.
 The following procedure can be used to produce such pages:
 \begin{enumerate}[(iii)]\listsize
  \item Use the \verb"table*" or \verb"figure*" environments in your
        document to create the space for your table or figure on the
        appropriate page of your document.
         Include the caption in this environment to ensure the correct
        numbering of floats and the correct entry in the `List of
        Illustrations' or `List of Tables'.
  \item Create a separate document with the corresponding document class
        but also with the \verb"landscape" document class option:
\begin{verbatim}
  \documentclass[landscape]{cupbook}
\end{verbatim}
  \item Include your complete tables and illustrations (or space for
        these) with captions using the \verb"table*" and \verb"figure*"
        environments.
  \item Before each float environment, use the \verb"\setcounter"
        command to ensure the correct numbering of the caption.
         For example,
\begin{verbatim}
 \setcounter{chapter}{3}
 \setcounter{table}{0}
 \begin{table*}
  \caption{The Largest Optical Telescopes}
  \label{tab1}
  \begin{tabular}{llllcll}
   \hline\hline
    :
  \end{tabular}
\end{table*}

\clearpage
\setcounter{figure}{12}
\begin{figure*}
 \vspace{20pc}
 \caption{Chart for a cold plasma}
 \label{fig13}
\end{figure*}
\end{verbatim}
 \end{enumerate}

\section{Additional facilities}

In addition to all the standard \LaTeX\ design elements, the CUPBOOK
class includes the following features:
 \begin{itemize}\listsize
  \item \verb"table*" and \verb"figure*" modified for wide tables and
        long captions; see section \ref{land}.
  \item \verb"proof" and \verb"proof*" environments.
  \item Control of enumerated lists with an optional argument for the
        widest label.
  \item \verb"theauthorindex" and \verb"thesubjectindex" environments
        (in addition to the standard \verb"theindex" environment) for
        author and subject indexes respectively. Authors using the CAPS
        option have an additional \verb"theobjectindex" environment.
  \item An \verb"\endnote" command for specifying an endnote,
        and \verb"\theendnotes" command for listing all the endnotes.
  \item \verb"exercises" and \verb"exerciselist" environments.
  \item \verb"thereferences" and \verb"thebibliography" (standard)
        environments for both a list of references and a bibliography.
  \item An extended \verb"\title" command which takes an optional argument
        for use as a \emph{subtitle} on the title page.
 \end{itemize}
 Also for multi-author options only:
 \begin{itemize}\listsize
  \item An extended \verb"\author" command
        which takes an optional argument for use in the running head
        and in the table of contents entry.
  \item An \verb"abstract" environment for use at the beginning of a chapter.
  \item A \verb"listofcontributors" environment which generates an unnumbered
        chapter, a table of contents entry and a list of contributors.
 \end{itemize}
 Once you have used these additional facilities in your document,
do not process it with a standard \LaTeX\ class file.

\subsection{Proofs}
\newtheorem{lemma}{Lemma}[section]

The \verb"proof" environment has been added to the standard \LaTeX\
constructs to provide a consistent format for proofs.
 For example,
 \begin{verbatim}
\begin{lemma}
  If \textbf{PP}$_0 \vdash$ a:A, then there exists a closed
  $\lambda$-term $a'$ s.t.
  \[
    \mathbf{PN}_0 \vdash a':A.
  \]
  Conversely, if \textbf{PN}$_0 \vdash$ a:A, then there exists
  a closed SK-term a$^\circ$ s.t.
  \[
    \mathbf{PP}_0 \vdash a^\circ:A.
  \]
\end{lemma}
\begin{proof}
  Use $K_\lambda$ and $S_\lambda$ to translate combinators
  into $\lambda$-terms. For the converse, translate
  $\lambda x$ \ldots by [$x$] \ldots and use induction
  and the lemma.
\end{proof}
\end{verbatim}
produces the following text:
 \begin{lemma}
  If \textbf{PP}$_0 \vdash$ a:A, then there exists a closed $\lambda$-term
  $a'$ s.t.
  \[
   \mathbf{PN}_0 \vdash a':A.
  \]
  Conversely, if \textbf{PN}$_0 \vdash$ a:A, then there exists a closed
  SK-term a$^\circ$ s.t.
  \[
   \mathbf{PP}_0 \vdash a^\circ:A.
  \]
 \end{lemma}
 \begin{proof}
  Use $K_\lambda$ and $S_\lambda$ to translate combinators into
  $\lambda$-terms. For the converse, translate $\lambda x$ \ldots by
  [$x$] \ldots and use induction and the lemma.
 \end{proof}
 The final \proofbox\ will not be included if the \verb|proof*|
environment is used.

\subsection{Lists}

The CUPBOOK class provides the three standard list environments:
 \begin{itemize}\listsize
  \item Numbered lists, created using the \verb"enumerate" environment.
  \item Bulleted lists, created using the \verb"itemize" environment.
  \item Labelled lists, created using the \verb"description" environment.
 \end{itemize}
 The \verb"enumerate" environment numbers each list item with an arabic
numeral; alternative styles can be achieved by inserting a redefinition
of the number labelling command after the \verb"\begin{enumerate}".
 For example, a list numbered with lower-case letters inside parentheses
can be produced by the following commands:
 \begin{verbatim}
    \begin{enumerate}\listsize
     \renewcommand{\theenumi}{(\alph{enumi})}
     \item first item
           :
    \end{enumerate}
\end{verbatim}
 This produces the following list:
 \begin{enumerate}\listsize
  \renewcommand{\theenumi}{(\alph{enumi})}
  \item first item
  \item second item
  \item etc.
 \end{enumerate}
% In the last example, the labels were pushed out into the margin
%because the standard list indention is designed to be sufficient for
%arabic numerals rather than the longer roman numerals.
 In the last example, the margin is larger than necessary
because the standard list indention is designed to be sufficient for
the longer roman numerals rather than the alphabetic labels used here.
 In order to enable different types of label to be used more easily, the
\verb"enumerate" environment in the CUPBOOK class can be given an
optional argument which (like a standard \verb"thebibliography"
environment) specifies the \emph{widest label}.
 For example,
 \begin{enumerate}[(iii)]\listsize
  \item first item
  \item second item
  \item etc.
 \end{enumerate}
 was produced by the following input:
 \begin{verbatim}
    \begin{enumerate}[(iii)]\listsize
     \item first item
           :
    \end{enumerate}
\end{verbatim}

\subsection{Indexes}

The standard \verb"theindex" environment is provided in the CUPBOOK
class file but two extra environments are also available and can be 
used in a similar way to \verb"theindex" environment:
\verb"theauthorindex" and \verb"thesubjectindex"
generate an `Author Index' and a `Subject Index' respectively.
Authors using the CAPS option have the additional \verb"\theobjectindex" environment which generates an `Object Index'. CAPS' authors
also have the facility to put some text across the two columns, 
by using the environment \verb"\begin{chaptertext} ... \end{chaptertext}", 
as shown in the following example:
\begin{verbatim}
  \begin{chaptertext}
    Chapter numbers given in \textbf{boldface} indicate that 
    the chapter was written by the listed author.
  \end{chaptertext}
  \begin{theauthorindex}
     :
\end{verbatim}
All these index environments generate an unnumbered chapter whose
title appears in the running heads and the table of contents
automatically.

\subsection{Endnotes}

In addition to footnotes, the CUPBOOK class provides a similar
facility for endnotes.
 A list of these endnotes can be produced in the form of an unnumbered
chapter at the end of the book (or an unnumbered section at the end of
each chapter with the multi-author options).

Endnotes can be entered into the text, at the place of reference, in a
similar way to footnotes but using the \verb"\endnote" command;
for example,
\begin{verbatim}
 ...this is some text\endnote{This is
             the text of the endnote.}
 and some more text...
\end{verbatim}
 Endnotes are referenced in the text with a superior number.

With the single-author options, the endnotes should be printed at the
end of the book, after the appendices but before the bibliography
and/or references.
 \begin{verbatim}
    :
 \theendnotes
 \begin{thebibliography}{xxx}
    :
\end{verbatim}
The \verb"\theendnotes" command generates an unnumbered
chapter which appears in the table of contents and enters section
headings with chapter names or numbers as appropriate.

With the multi-author option, the endnotes should be printed at the end
of the chapter using the same \verb"\theendnotes" command.

\subsection{Exercises}

Two environments, \verb"exercises" and \verb"exerciselist", are provided to
generate lists of exercises in the required format.
 The \verb"exercise" environment starts a new unnumbered section, resets the
equation numbering and starts an exercise list; this should be sufficient for
most normal requirements. For example,
 \begin{verbatim}
\begin{exercises}
 \item Let $M$ be the unit circle in the complex plane.
       Explain how the map $\pi:\mathbf{R} \rightarrow S^1$
       given by $x\mapsto\exp(\mathrm{i}x)$ can be used to
       define a set of charts on the circle $S^1$.
 \item ...
\end{exercises} % end of enumeration of exercise
\end{verbatim}
 If more control is needed (e.g. if exercises occur throughout a chapter)
the \verb"exerciselist" environment can be used to generate the list alone.

\subsection{References and Bibliographies}

Some books may need both a list of references and a bibliography. The CUPBOOK
class includes an additional environment, \verb"thereferences", which can be
used to generate a list of references. The standard \verb"thebibliography"
environment can be used to produce a bibliography.
 The following listing shows some references prepared in the
appropriate style; this produces the reference list on page~\pageref{reflist}.
 \begin{verbatim}
\begin{thereferences}{99}
 \bibitem{abbott}
         Abbott, L.F. and Deser, S. (1982).
         Stability of gravity with a cosmological constant,
         \textit{Nucl. Phys.} \textbf{B195}, 76--96.
 \bibitem{adams}
         Adams, J.F. (1981).
         Spin (8), triality, $F_4$ and all that, in
         \textit{Superspace and Supergravity},
         ed. S.W.~Hawking and M.~R\"ocek
         (Cambridge University Press, Cambridge).
 \bibitem{arnold}
         Arnol'd, V.I. (1978).
         \textit{Mathematical Methods of Classical Mechanics}
         (Springer, New York).
 \bibitem{buch}
         Buchdahl, N.P. (1982).
         Applications of Several Complex Variables to
         Twistor Theory, Oxford University D. Phil. thesis.
\end{thereferences}
\end{verbatim}
 Both environments generate an unnumbered chapter whose
title appears in the running heads and the table of contents
automatically.

\subsection{Subtitles}

The standard \verb"\title" command has been extended to take an optional
argument which is then used as a subtitle on the main title page.
For example, this document has the following title command:
\begin{verbatim}
\title[\LaTeXe\ Style Guide for Authors]
      {CUP Standard Designs}
\end{verbatim}

\subsection{Author's name and affiliation}

In the CUPBOOK multi-author class options (\verb"*B"), the title of the
chapter and the author's name (or authors' names) are used both at the
beginning of the chapter and throughout the chapter as running headlines at the
top of every page. The chapter heading is used on odd-numbered pages (rectos)
and the author's name appears on even-numbered pages (versos). Although the
main chapter heading can run to several lines of text, the running head line
must be a single line. Moreover, the chapter heading can incorporate new line
commands (e.g.\ \verb"\\") but these are not acceptable in a running headline.

The standard \verb"\chapter" command has an optional argument which is used to
generate a table of contents entry and the running headline.
To enable you to specify an alternative short form of the author's name, the
standard \verb"\author" command has been extended to take an optional argument
to be used as the running headline:
\begin{verbatim}
    \author[Author's name]{The full author's name with
       affiliation if necessary
      \and
       additional authors' names and affiliations}
\end{verbatim}

\section{Some guidelines for using standard facilities}

The following notes may help you achieve the best effects with the
CUPBOOK class file.

\subsection{Sections}

\LaTeX\ provides five levels of section headings and they are all
defined in the CUPBOOK class file:
\begin{itemize}\listsize
  \item Heading A -- \verb"\section".
  \item Heading B -- \verb"\subsection".
  \item Heading C -- \verb"\subsubsection".
  \item Heading D -- \verb"\paragraph".
  \item Heading E -- \verb"\subparagraph".
\end{itemize}
 Numbers are given for section, subsection and subsubsection headings,
but only section and subsection headings appear in the table of contents.

Authors using the CAPS option have an extra two headings:
\begin{itemize}\listsize
  \item Heading F -- \verb"\xhead".
  \item Heading G -- \verb"\yhead".
\end{itemize}

\subsection{Page styles and running headlines}

In CUPBOOK, as in BOOK class, chapter titles and section headings are
used as running headlines at the top of every page.
 The section heading is used on odd-numbered pages (rectos) and the
chapter title appears on even-numbered pages (versos).

The \verb"\pagestyle" and \verb"\thispagestyle" commands should
\emph{not} be used.
 Similarly, the commands \verb"\markright" and \verb"\markboth" should
not be\break necessary.

With the multi-author option, the title of the chapter and the author's
name (or authors' names) are used both at the beginning of the chapter
for the main heading and throughout the chapter as running headlines at
the top of every page.
 The title is used on odd-numbered pages (rectos) and the author's name
appears on even-numbered pages (versos).

With both single-author and multi-author options, the main heading can
run to several lines in display but the running headline must be a
single line.
 Moreover, the main heading can also incorporate new line commands
(e.g. \verb"\\") but these are not acceptable in a running headline.

 An alternative short chapter title can be specified using the optional
argument on the \verb"\chapter" command.
 To enable you to specify an alternative short author's name, the
standard \verb"\author" command has been extended to take an optional
argument to be used as the running headline and added to the short
chapter title as the table of contents entry:
 \begin{verbatim}
  \author[M.P. Seah]
         {Michael P. Seah \\ \Department of Texology,
          University of Cambridge.}
 \chapter[Mixtures of distributions]
         {Mixtures of exponential and other continuous
          distributions}
\end{verbatim}
The \verb"\author" command should appear before the \verb"\chapter"
command to which it refers, but after the \verb"maketitle" command.

\subsection{Illustrations (or figures)}

The CUPBOOK class will cope with most positioning of your
illustrations and you should not normally use the optional positional
qualifiers on the \verb"figure" environment which would override these
decisions.
 Figure captions should be below the figure itself, therefore the
\verb"\caption" command should appear after the figure or space left
for an illustration.
 For example, Figure~\ref{sample-figure} is produced using the
following commands:
 \begin{verbatim}
  \begin{figure}
    \centering
    \vspace{4cm}
    \caption{An example figure with space for artwork.}
    \label{sample-figure}
  \end{figure}
\end{verbatim}
  \begin{figure}
    \centering
    \vspace{4cm}
    \caption{An example figure with space for artwork.}
    \label{sample-figure}
  \end{figure}

\subsection{Tables}

The CUPBOOK class will cope with most positioning of your tables and
you should not normally use the optional positional qualifiers on the
\verb"table" environment which would override these decisions.
 Table captions (titles) must be at the top, therefore the
\verb"\caption" command should appear before the body of the table.
 For example, Table~\ref{sample-table} is produced using the following
commands:
 \begin{verbatim}
 \begin{table}
  \caption{An example table.}
    \begin{tabular}{c|ccc}
     \hline \hline
     {Figure} & {$hA$} & {$hB$} & {$hC$}\\
     \hline
     2 & $\exp\left(\pi i\frac58\right)$
       & $\exp\left(\pi i\frac18\right)$ & $0$\\
     3 & $-1$    & $\exp\left(\pi i\frac34\right)$ & $1$\\
     4 & $-4+3i$ & $-4+3i$ & $\frac74$\\
     5 & $-2$    & $-2$    & $\frac54 i$ \\
     \hline \hline
    \end{tabular}
  \label{sample-table}
\end{table}
\end{verbatim}
 \begin{table}
  \caption{An example table.}
    \begin{tabular}{c|ccc}
     \hline \hline
     {Figure} & {$hA$} & {$hB$} & {$hC$}\\
     \hline
     2 & $\exp\left(\pi i\frac58\right)$
       & $\exp\left(\pi i\frac18\right)$ & $0$\\
     3 & $-1$    & $\exp\left(\pi i\frac34\right)$ & $1$\\
     4 & $-4+3i$ & $-4+3i$ & $\frac74$\\
     5 & $-2$    & $-2$    & $\frac54 i$ \\
     \hline \hline
    \end{tabular}
  \label{sample-table}
 \end{table}
 The \verb"tabular" environment should be used to produce ruled tables;
it has been modified for the CUP styles in the following ways:
 \begin{enumerate}[(ii)]\listsize
  \item Additional vertical space is inserted on either side of a rule
        (produced by \verb"\hline");
  \item Vertical lines are not produced.
 \end{enumerate}
 Commands to redefine quantities such as \verb"\arraystretch" should be
omitted.

\begin{thereferences}{99}
 \label{reflist}
  \bibitem{abbott}
  Abbott, L.F. and Deser, S. (1982). Stability of gravity with a
  cosmological constant, \textit{Nucl. Phys.} \textbf{B195}, 76--96.

  \bibitem{adams}
  Adams, J.F. (1981). Spin (8), triality, $F_4$ and all that, in
  \textit{Superspace and Supergravity}, ed. S.W.~Hawking and M.~R\"ocek
  (Cambridge University Press, Cambridge).

  \bibitem{arnold}
  Arnol'd, V.I. (1978). \textit{Mathematical Methods of Classical
  Mechanics} (Springer, New York).

  \bibitem{buch}
  Buchdahl, N.P. (1982). Applications of Several Complex Variables to
  Twistor Theory, Oxford University D. Phil. thesis.
\end{thereferences}

\cleardoublepage
\appendix
\chapter{Notes for Editors}

\section{Production of crc}

This appendix contains additional information which may be useful to
those who are involved with the final production stages of a book.

\subsection{Prelims}
 \begin{itemize}\listsize
  \item Title pages in \textit{jobname\tt.TTL}
  \item In the CAPS design, \verb"\author" and \verb"\title" must be 
        letterspaced. See Lewin for pattern copy
  \item \verb"\halftitle" command to use in \textit{jobname\tt.TTL}
  \item \texttt{indented} environment for use in \textit{jobname\tt.TTL}
  \item Problems with footnotes at the top of a page
  \item NB. optional argument to \texttt{enumerate} environment to set the
        left margin equal to the widest label
%%%  \item Fonts -- \verb"\itbf"
%%%  \item Large fonts -- substitute correct sizes for \verb"\LARGE" and
%%%        \verb"\huge".
\end{itemize}

\subsection{Footnotes.}
If a footnote falls at the bottom of a page, it is possible for
the footnote to appear on the following page (a feature of \TeX ).
Check for this.

\subsection{Table footnotes and long captions.}
Set the table in a minipage that has the same width as the table body.
The caption will be set to the same width as the table and the footnotes
will fall at the bottom of the minipage.

\subsubsection{Wide tables -- more than 2/3 measure.}
Set with \verb"table*", giving full-out rules. Centre the body with
\verb"@{\hspace{"\textit{wd}\verb"}" in \textit{cols} argument of the
\verb"\begin{tabular}" command; \textit{wd} is horizontal space required
to centre the table. Only do in Times.


\section{Font sizes}

The CUPBOOK class file defines all the standard \LaTeX\ font sizes:
\begin{itemize}\listsize
  \item \verb"\tiny" -- {\tiny This is tiny text. }
  \item \verb"\scriptsize" -- {\scriptsize This is scriptsize text.}
  \item \verb"\footnotesize" -- {\footnotesize This is footnotesize text.}
  \item \verb"\small" -- {\small This is small text.}
  \item \verb"\normalsize" -- This is normalsize text (default).
  \item \verb"\large" -- {\large This is large text.}
  \item \verb"\Large" -- {\Large This is Large text.}
  \item \verb"\LARGE" -- {\LARGE This is LARGE text.}
  \item \verb"\huge" -- {\huge This is huge text.}
  \item \verb"\Huge" -- {\Huge This is Huge text.}
\end{itemize}
The CUPBOOK class also defines some non-standard font sizes:
\begin{itemize}\listsize
  \item \verb"\smallish" -- {\smallish This is the text size used for
        some displayed text environments.}
  \item \verb"\enotesize" -- {\enotesize This is the text size used for
        endnotes.}
  \item \verb"\normalsmall" -- {\normalsmall Used in prelims;
        e.g. list of illustrations.}
  \item \verb"\listsize" -- {\listsize This is the text size used for
        lists.}
\end{itemize}
The CAPS design defines further fonts;
all the sizes are summarized in Table~\ref{tab:fontsizes}.
\begin{table*}[hbt]
\caption{Type sizes for \LaTeX\ size-changing commands.}
\label{tab:fontsizes}
 \begin{tabular}{l|r@{/}lr@{/}lr@{/}lp{4cm}}
 \hline\hline
  \multicolumn{1}{l|}{\it Size}&\multicolumn{6}{c|}{\it size/baseline} & \it Usage\\
& \multicolumn{2}{c}{\it Large options} & \multicolumn{2}{c}{\it Small options} & \multicolumn{2}{c}{\it caps option} & \\
 \hline
 \verb"\tiny"         &  5pt &  6pt &  5pt &  6pt & 5pt & 6pt & -- \\
 \verb"\scriptsize"   &  7pt &  8pt &  7pt &  8pt & 7pt &  8pt & -- \\
 \verb"\footnotesize" &  8pt &  9pt &  8pt &  9pt & 8pt &  9pt & footnotes,
                   indexes, author affiliation in title \\
 \verb"\affilsize"    &  n   &  a   &  n   & a    & 8pt & 10pt &
                   affiliations in prelims\\
 \verb"\refsize"      &  n   &  a   &  n   & a    & 8.5pt & 10pt &
                   references\\ 
 \verb"\enotesize"    &  9pt & 10pt &  9pt & 10pt & 9pt & 10pt & endnotes,
                   author affiliation in chapter opening \\
 \verb"\small"        & 10pt & 11pt &  9pt & 10pt & 9pt & 10pt & tables,
                   references\footnote{Except in CAPS design}, bibliography,
                   appendices, figure captions\\
 \verb"\smallish"     & 10pt & 12pt &  9pt & 11pt & 9pt & 11pt & verse,
                   quote$^a$, quotation environments\\
 \verb"\quotesize"    &  n   &  a   &  n   & a    & 9.5pt & 12pt &
                   quote environment, lists\\ 
 \verb"\normalsmall"  & 10pt & 12pt & 10pt & 12pt & 10pt & 12pt & preliminary
                   material\\
 \verb"\listsize"     & 10pt & 13pt & 10pt & 13pt & 10pt & 12.5pt & lists$^a$ \\
 \verb"\normalsize"   & 11pt & 14pt & 10pt & 13pt & 10pt & 12.5pt & main text
                   size, section headings$^a$, etc.\\
 \verb"\sectionsize"  &  n   &  a   &  n   & a    & 11pt & 12.5pt &
                   section headings\\ 
 \verb"\large"        & 11pt & 14pt & 11pt & 14pt & 11pt & 14pt & half title,
                   author's name on title page\\
 \verb"\Large"        & 12pt & 17pt & 12pt & 17pt & 12pt & 17pt & sub title
                   on title page\\
 \verb"\chaptitlesize"  
                      &  n   &  a   &  n   & a    & 20pt & 25pt &
                   chapter titles\\ 
 \verb"\LARGE"        & 17pt & 19pt & 17pt & 19pt & 20pt & 25pt & part and
                   chapter titles, main headings (\emph{should be} 16/19pt)\\
 \verb"\huge"         & 20pt & 24pt & 20pt & 24pt & 24pt & 25pt & title on
                   title page, part and chapter numbers (\emph{should be}
                   18/24pt)\\
 \verb"\Huge"         & 17pt & 24pt & 17pt & 24pt & 24pt & 25pt & -- \\
 \verb"\chapnumsize"  
                      &  n   &  a   &  n   & a    & 24pt & 25pt &
                   chapter number size\\ 
 \hline\hline
 \end{tabular}
\end{table*}

\end{document}
%
% end of cup2egui.tex

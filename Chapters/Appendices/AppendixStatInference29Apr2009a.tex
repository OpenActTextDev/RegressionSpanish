
\appendix
\chapter{Basic Statistical Inference}

\vspace{-1.5in}

{\small \textit{Appendix Preview.} This appendix provides
definitions and facts from a course in basic statistical inference
that are needed in one's study of regression analysis.}

\bigskip

\section{Distributions of Functions of Random
Variables}\index{statistic}\index{distributions!sampling}

\textbf{Statistics and Sampling Distributions.} A \emph{statistic}
summarizes information in a sample and hence is a function of
observations $ y_1,\ldots,y_n$. Because observations are
realizations of random variables, the study of distributions of
functions of random variables is really the study of the
distributions of statistics, known as \emph{sampling distributions}.
Linear combinations of the form $\sum_{i=1}^n a_i y_i$ represent an
important type of function. Here,  $a_1,\ldots,a_n$ are known
constants. To begin, we suppose that $y_1,\ldots,y_n$ are mutually
independent random variables with $\mathrm{E~}y_i = \mu_i$ and
$\mathrm{ Var~}y_i = \sigma_i^2.$ Then, by the \emph{linearity of
expectations}, we have
\begin{equation*}
\mathrm{E}\left( \sum_{i=1}^n a_i y_i \right) = \sum_{i=1}^n a_i
\mu_i~~~\mathrm{and}~~~\mathrm{Var}\left( \sum_{i=1}^n a_i y_i
\right) = \sum_{i=1}^n a_i^2 \sigma_i^2.
\end{equation*}
An important theorem in mathematical statistics is that, if each
random variable is normally distributed, then linear combinations
are also normally distributed. That is, we
have:\index{theorems!linearity of normal random variables}

\textbf{Linearity of Normal Random Variables}. Suppose that
$y_1,\ldots,y_n$ are mutually independent random variables with
$y_i\sim N(\mu_i,\sigma_i^2)$. (Read \ ``$\sim $'' to mean ``is
distributed as.'') Then,
\begin{equation*}
\sum_{i=1}^n a_i y_i \sim N\left( \sum_{i=1}^n a_i \mu_i,
\sum_{i=1}^n a_i^2 \sigma_i^2 \right) .
\end{equation*}
There are several applications of this important property. First, it
can be checked that if $y \sim N(\mu ,\sigma ^2)$, then $(y - \mu)/
\sigma \sim N(0,1)$. Second, assume that $y_1,\ldots,y_n$ are
identically and independently distribution (\emph{i.i.d.}) as
$N(\mu, \sigma ^2)$ and take $a_i=n^{-1}$. Then, we have
\begin{equation*}
\overline{y} = \frac{1}{n}\sum_{i=1}^n y_i \sim N\left( \mu
,\frac{\sigma ^2}{n}\right) .
\end{equation*}
Equivalently, $\sqrt{n}\left( \overline{y}-\mu \right) /\sigma $ is standard
normal.

Thus, the important sample statistic $\overline{y}$\ has a normal
distribution. Further, the distribution of the sample variance
$s_y^2$ can also be calculated. For $y_1,\ldots,y_n$ are i.i.d.
$N(\mu ,\sigma ^2) $, we have that $\left( n-1\right) s_y^2 /
\sigma^2\sim \chi_{n-1}^2, $ a $\chi ^2$ (chi-square) distribution
with $n$-1 degrees of freedom. Further, $\overline{y}$\ is
independent of $s_y^2$. From these two results, we have that
\begin{equation*}\index{distributions!chi-square}
\frac{\sqrt{n}}{s_y}\left( \overline{y}-\mu \right) \sim t_{n-1},
\end{equation*}
a \textit{t}-distribution with $n$-1 degrees of
freedom.\index{distributions!t-@{$t-$}}\index{distributions!chi-square}

\section{Estimation and Prediction}\index{estimator}

Suppose that $y_1,\ldots,y_n$ are i.i.d. random variables from a
distribution that can be summarized by an unknown parameter $\theta
$. We are interested in the quality of an estimate of $\theta $ and
denote $ \widehat{\theta }$\ as this estimator. For example, we
consider $\theta =\mu $ with $\widehat{\theta }=\overline{y}$ and
$\theta =\sigma ^2$ with $ \widehat{\theta }=s_y^2$\ as our leading
examples.

\index{estimator!point}\index{estimator!unbiased}\index{estimator!consistency}

\textbf{Point Estimation and Unbiasedness}. Because $\widehat{\theta
}$\ provides a (single) approximation of $\theta $, it is referred
to as a \emph{point estimate} of $\theta $. As a statistic,
$\widehat{\theta }$\ is a function of the observations
$y_1,\ldots,y_n$ that varies from one sample to the next. Thus,
values of $\widehat{\theta }$\ vary from one sample to the next. To
examine how close $\widehat{\theta }$\ tends to be to $\theta $ , we
examine several properties of $\widehat{\theta }$, in particular,
the \emph{bias} and \emph{consistency}. A point estimator
$\widehat{\theta }$\ is said to be an \emph{unbiased estimator} of
$\theta $ if $\mathrm{E~} \widehat{\theta }=\theta $. For example,
since $\mathrm{E~}\overline{y}=\mu $ , $\overline{y}$\ is an
unbiased estimator of $\mu $.

\textbf{Finite Sample versus Large Sample Properties of Estimators}.
Biasedness is said to be a \emph{finite sample} property since it is
valid for each sample size $n$. A \emph{limiting}, or \emph{large,
sample} property is \emph{consistency}. Consistency is expressed in
two ways, \emph{weak} and \emph{strong}, consistency. An estimator
is said to be \emph{(weakly) consistent} if
\begin{equation*}
\lim_{n\rightarrow \infty }\Pr \left( |\widehat{\theta }-\theta |<h\right)
=1,
\end{equation*}
for each positive $h$. An estimator is said to be \emph{strongly consistent}
if $\lim_{n\rightarrow \infty }~\widehat{\theta }=\theta $, with probability
one.

\index{least squares!principle}

\textbf{Least Squares Estimation Principle}. In this text, two main
estimation principles are used, \emph{least squares} estimation and
\emph{maximum likelihood} estimation. For the least squares
procedure, consider independent random variables  $y_1,\ldots,y_n$
with means $\mathrm{E~}y_i= \mathrm{g}_i(\theta )$. Here,
$\mathrm{g}_i(.)$ is a known function up to $\theta $, the unknown
parameter. The least squares estimator is that value of $\theta $
that minimizes the sum of squares
\begin{equation*}
\mathrm{SS}(\theta )=\sum_{i=1}^n\left( y_i-\mathrm{g}_i(\theta
)\right) ^2.
\end{equation*}

\index{maximum likelihood estimation}

\textbf{Maximum Likelihood Estimation Principle}. Maximum likelihood
estimates are values of the parameter that are ``most likely'' to
have been produced by the data. Consider the independent random
variables $y_1,\ldots,y_n$ with probability function $
\mathrm{f}_i(a_i,\theta )$. Here, $\mathrm{f}_i(a_i,\theta )$ is
interpreted to be a probability mass function for discrete $y_i$ or
a probability density function for continuous $y_i$, evaluated at
$a_i$, the realization of $y_i$. The function
$\mathrm{f}_i(a_i,\theta )$ is assumed known up to $\theta $, the
unknown parameter. The likelihood of the random variables
$y_1,\ldots,y_n$ taking on values $a_1,\ldots,a_n$ is
\begin{equation*}
\mathrm{L}(\theta )=\prod\limits_{i=1}^n \mathrm{f}_i(a_i,\theta ).
\end{equation*}

The value of $\theta $ that maximizes $\mathrm{L}(\theta )$ is
called the \emph{maximum likelihood
estimator}.\index{estimator!maximum likelihood}\index{likelihood
inference!maximum likelihood estimation}

\textbf{Confidence Intervals}. Although point estimates provide a
single approximation to parameters, \emph{interval estimates}
provide a range that include parameters with a certain prespecified
level of probability, or \emph{confidence}. A pair of statistics,
$\widehat{\theta }_1$ and $ \widehat{\theta }_{2}$,\ provide an
interval of the form $\left[ \widehat{ \theta }_1<\widehat{\theta
}_{2}\right] $. This interval is a $ 100(1-\alpha )\%$ confidence
interval for $\theta $ if\index{estimator!interval|see{confidence
interval}}
\begin{equation*}
\Pr \left( \widehat{\theta }_1<\theta <\widehat{\theta }_{2}\right)
\geq 1-\alpha .
\end{equation*}
For example, suppose that $y_1,\ldots,y_n$ are i.i.d. $N(\mu ,\sigma
^2)$ random variables. Recall that $\sqrt{n}\left( \overline{y}-\mu
\right) /s_y\sim t_{n-1}$. This fact allows us to develop a
$100(1-\alpha )\%$ confidence interval for $\mu $ of the form
$\overline{y}\pm (t-value)s_y/ \sqrt{n}$, where $t-value$ is the
$(1-\alpha /2)^{th}$ percentile from a \textit{t}-distribution with
$n$-1 degrees of freedom.

\textbf{Prediction Intervals}. Prediction intervals have the same
form as confidence intervals. However, a confidence interval
provides a range for a parameter whereas a prediction interval
provides a range for external values of the observations. Based on
observations $y_1,\ldots,y_n$, we seek to construct statistics
$\widehat{\theta }_1$ and $\widehat{\theta }_{2}$ such that \newline
$\Pr \left( \widehat{\theta }_1<y^{\ast }<\widehat{\theta
}_{2}\right) \geq 1-\alpha .$ Here, $y^{\ast }$ is an additional
observation that is not a part of the sample.

\section{Testing Hypotheses}\index{hypothesis test}\index{hypothesis
test!null hypothesis}\index{hypothesis test!alternative hypothesis}
\index{hypothesis test!one-sided alternative}\index{hypothesis
test!two-sided alternative}

\textbf{Null and Alternative Hypotheses and Test Statistics}. An
important statistical procedure involves verifying ideas about
parameters. That is, before the data are observed, certain ideas
about the parameters are formulated. In this text, we consider a
\emph{null hypothesis} of the form $ H_0:\theta =\theta_0$ versus an
\emph{alternative hypothesis}. We consider both a \emph{two-sided
alternative}, $H_{a}:\theta \neq \theta_0$ , and \emph{one-sided
alternatives}, either $H_{a}:\theta >\theta_0$ or $ H_{a}:\theta
<\theta_0$. To choose between these competing hypotheses, we use a
test statistic $T_n$ that is typically a point estimate of $\theta $
or a version that is rescaled to conform to a reference distribution
under $ H_0$. For example, to test $H_0:\mu =\mu_0$, we often use
$T_n= \overline{y}$ or $T_n=\sqrt{n}\left( \overline{y}-\mu_0\right)
/s_y$. Note that the latter choice has a $t_{n-1}$ distribution,
under the assumptions of i.i.d. normal data.

\index{hypothesis test!rejection region}\index{hypothesis
test!critical region}\index{hypothesis test!Type I
error}\index{hypothesis test!Type II error}

\textbf{Rejection Regions and Significance Level}. With a statistic
in hand, we now establish a criterion for deciding between the two
competing hypotheses. This can be done by establishing a
\emph{rejection}, or \emph{critical, region}. The critical region
consists of all possible outcomes of $ T_n$ that leads us to reject
$H_0$ in favor of $H_{a}$. In order to specify the critical region,
we first quantify the types of errors that can be made in the
decision making procedure. A \emph{Type I error} consists of
rejecting $H_0$ falsely and a \emph{Type II error} consists of
rejecting $H_{a}$ falsely. The probability of a Type I error is
called the \emph{significance level}. Prespecifying the significance
level is often enough to determine the critical region. For example,
suppose that $y_1,\ldots,y_n$ are i.i.d. $N(\mu ,\sigma ^2)$ and we
are interested in deciding between $ H_0:\mu =\mu_0$ and $H_{a}:\mu
>\mu_0$. Thinking of our test statistic $T_n=\overline{y}$, we
know that we would like to reject $H_0$ if $\overline{y}$\ is larger
that $\mu_0$. The question is how much larger? Specifying a
significance level $\alpha $, we wish to find a critical region of
the form $\{\overline{y}>c\}$ for some constant $c$. To this end, we
have

\begin{eqnarray*}
\alpha  &\mathrm{=}&\Pr \mathrm{(Type~I~error)=}\Pr \mathrm{(}\
\mathrm{Reject~}H_0\mathrm{~assuming~} H_0:\mu =\mu_0
\mathrm{~is~true)} \\
&=&\Pr (\overline{y}>c)=\Pr (\sqrt{n}\left( \overline{y}-\mu
_0\right)
/s_y>\sqrt{n}\left( c-\mu_0 \right) /s_y) \\
&=&\Pr (t_{n-1}>\sqrt{n}\left( c-\mu_0 \right) /s_y).
\end{eqnarray*}

With $df=n-1$ degrees of freedom, we have that $t-value = \sqrt{n}
\left( c-\mu_0\right) /s_y$ where the $t-value$ is the
$(1-\alpha)^{th}$ percentile from a \textit{t}-distribution. Thus,
solving for $c$, our critical region is of the form $\{\overline{y}
> \mu_0 + (t-value)/s_y/\sqrt{n}\}$.

\textbf{Relationship between Confidence Intervals and Hypothesis
Tests}. Similar calculations show, for testing $H_0:\mu = \mu_0$
versus $H_{a}:\theta \neq \theta_0$, that the critical region is of
the form $\{ \overline{y} > \mu_0 + (t-value)/s_y/\sqrt{n} ~
\mathrm{or~}\overline{y} > \mu _0 + (t-value)/s_y/\sqrt{n}\}$. Here,
the $t$-value is a $(1-\alpha /2)^{th}$ percentile from a
\textit{t}-distribution with $df=n-1$ degrees of freedom. It is
interesting to note that the event of falling in this two-sided
critical region is equivalent to the event of $\mu_0$ falling
outside the confidence interval $\overline{y}\pm
(t-value)s_y/\sqrt{n}$. This establishes the fact that confidence
intervals and hypothesis tests are really reporting the same
evidence with different emphasis on interpretation of the
statistical inference.

\index{hypothesis test!p-value@{$p$-value}}\index{symbols!$p$-value,
probability value}

\textit{p}\textbf{-value}. Another useful concept in hypothesis
testing is the \emph{p-value}, which is short hand for
\emph{probability value}. For a data set, a \textit{p}-value is
defined to be the smallest significance level for which the null
hypothesis would be rejected. The \textit{p}-value is a useful
summary statistic for the data analyst to report since it allows the
reader to understand the strength of the deviation from the null
hypothesis.


\chapter*{Dedication}


There is an old saying, attributed to Sir Issac Newton and that can
be found on web at Google Scholar,


\bigskip\bigskip

\begin{center}
``If I have seen far, it is by standing on the shoulders of
giants.''
\end{center}

\bigskip\bigskip

\noindent I dedicate this book to the memory of two giants who
helped me, and everyone who knew them, see farther and live better
lives:
\bigskip\bigskip


\begin{center}James C. Hickman\end{center}


and



\begin{center}Joseph P. Sullivan.\end{center}



\chapter*{Preface}

\vspace{-1in}

Actuaries and other financial analysts quantify situations using
data -- we are ``numbers'' people. Many of our approaches and models
are stylized, based on years of experience and investigations
performed by legions of analysts. However, the financial and risk
management world evolves rapidly. Many analysts are confronted with
new situations in which tried-and-true methods simply do not work.
This is where a toolkit like regression analysis comes in.

Regression is the study of relationships among variables. It is a
generic statistics discipline that is not restricted to the
financial world -- it has applications in fields of social,
biological and physical sciences. You can use regression techniques
to investigate large and complex data sets. To familiarize you with
regression, this book explores many examples and data sets based on
actuarial and financial applications. This is not to say that you
will not encounter applications outside of the financial world (for
example, an actuary may need to understand the latest scientific
evidence on genetic testing for underwriting purposes). However, as
you become acquainted with this toolkit, you will see how regression
can be applied in many (and sometimes new) situations.

\section*{Who Is This Book For?}

This book is written for financial analysts who face uncertain
events and wish to quantify these events using empirical
information. No industry knowledge is assumed although readers will
find the reading much easier if they have an interest in the
applications discussed here! This book is designed for students who
are just being introduced to the field as well as industry analysts
who would like to brush up on old techniques and (for the later
chapters) get an introduction to new developments.

To read this book, I assume knowledge comparable to a one semester
introduction to probability and statistics - Appendix A1 provides a
brief review to brush up if you are rusty. Actuarial students in
North America will have a one-year introduction to probability and
statistics - this type of introduction will help readers grasp
concepts more quickly than a one semester background. Finally,
readers will find matrix, or linear, algebra helpful, although not a
prerequisite for reading this text.

Different readers are interested in understanding statistics at
different levels. This book is written to accommodate the ``armchair
reader,'' that is, one who passively reads and does not get involved
by attempting the exercises in the text. Consider an analogy to
football, or any other game. Just like the armchair quarterback of
football, there is a great deal that you can learn about the game
just by watching. However, if you want to sharpen your skills, you
have to go out and play the game. If you do the exercises or
reproduce the statistical analyses in the text, you will become a
better player. Still, this text is written by interweaving examples
with the basic principles. Thus, even the armchair reader can obtain
a solid understanding of regression techniques through this text.

\section*{What Is This Book About?}

The Table of Contents provides an overview of the topics covered,
organized into four parts. The first part introduces linear
regression. This is the core material of the book, with refreshers
on mathematical statistics, distributions and matrix algebra woven
in as needed.

The second part is devoted to topics in times series. Why integrate
time series topics into a regression book? The reasons are simple,
yet compelling; most accounting, financial and economic data become
available over time. Although cross-sectional inferences are useful,
business decisions need to be made in real time with currently
available data. Chapters 7-10 introduce time series techniques that
can be readily accomplished using regression tools (and there are
many).

Nonlinear regression is the subject of the third part. Many modern
day ``predictive modeling'' tools are based on nonlinear regression
- these are the workhorses of statistical shops in the financial and
risk management industry.

The fourth part concerns ``actuarial applications,'' topics that I
have found relevant in my research and consulting work in financial
risk management. The first four chapters of this part consists of
variations of regression models that are particularly useful in risk
management. The last two chapters focus on communications,
specifically, report writing and designing graphs. Communicating
information is an important aspect of every technical discipline and
statistics is certainly no exception.


\section*{How Does This Book Deliver Its Message?}


\textbf{Chapter Development}. Each chapter has several examples
interwoven with theory. In chapters where a model is introduced, I
begin with an example and discuss the data analysis without regard
to the theory. This analysis is presented at an intuitive level,
without reference to a specific model. This is straightforward,
because it amounts to little more than curve fitting. The theme is
to have students summarize data sensibly without having the notion
of a model obscure good data analysis. Then, an introduction to the
theory is provided in the context of the introductory example. One
or more additional examples follow that reinforce the theory already
introduced and provide a context for explaining additional theory.
In Chapters 5 and 6, that do not introduce models but rather
techniques for analysis, I begin with an introduction of the
technique. This introduction is then followed by an example that
reinforces the explanation. In this way, the data analysis can be
easily omitted without loss of continuity, if time is a concern.

\textbf{Real Data}. Many of the exercises ask the reader to work
with real data. The need for working with real data is well
documented; for example, see Hogg (1972) or Singer and Willett
(1990). Some criteria of Singer and Willett for judging a good data
set include: (1) authenticity, (2) availability of background
information, (3) interest and relevance to substantive learning, and
(4) availability of elements with which readers can identify. Of
course, there are some important disadvantages to working with real
data. Data sets can quickly become outdated. Further, the ideal data
set to illustrate a specific statistical issue is difficult to find.
This is because with real data, almost by definition, several issues
occur simultaneously. This makes it difficult to isolate a specific
aspect. I particularly enjoy working with large data sets. The
larger the data set, the greater is the need for statistics to
summarize the information content.

\marginparjed{The larger the data set, the greater is the need for
statistics to summarize the information content.}


\textbf{Statistical Software and Data}. My goal in writing this text
is to reach a broad group of students and industry analysts. Thus,
to avoid excluding large segments, I chose not to integrate any
specific statistical software package into the text. Nonetheless,
because of the applications orientation, it is critical that the
methodology presented be easily accomplished using readily available
packages. For the course taught at the University of Wisconsin, I
use the statistical packages SAS and ``R.'' On the book web site,

\bigskip

{ \url{http://research.bus.wisc.edu/RegActuaries}}

\bigskip

\empexjed{TermLife}

\noindent users will find scripts written in SAS and R for the
analysis presented in the text. The data are available in text
format, allowing readers to employ any statistical packages that
they wish. When you see a display such as this in the margin, you
will also be able to find this data set (\emph{TermLife}) on the
book web site.

\textbf{Technical Supplements}. The technical supplements reinforce
and extend the results in the main body of the text by giving a more
formal, mathematical treatment of the material. This treatment is in
fact a supplement because the applications and examples are
described in the main body of the text. For readers with sufficient
mathematical background, the supplements provide additional material
that is useful in communicating to technical audiences. The
technical supplements provide a deeper, and broader, coverage of
applied regression analysis.

I believe that analysts should have an idea of ``what is going on
under the hood,'' or ``how the engine works.'' Most of these topics
will be omitted from the first reading of the material. However, as
you work with regression, you will be confronted with questions on
``Why?'' and you will need to get into the details to see exactly
how a certain technique works. Further, the technical supplements
provide a menu of optional items that an instructor may wish to
cover.


\textbf{Suggested Courses}. There is a wide variety of topics that
can go into a regression course. Here are some suggested courses.
The course that I teach at the University of Wisconsin is the first
on the list in the following table.

\begin{table}[h]
\scalefont{0.9}

\begin{tabular}{lll}
\hline
\multicolumn{1}{c}{Audience} & \multicolumn{1}{c}{Nature of Course}& \multicolumn{1}{c}{Suggested Chapters} \\
\hline One-year background in & Survey of regression and  &
 Chapters
1-8 ,11-13, 20-21,\\
~~~probability and statistics & ~~~time series models &
~~~main body of text only \\
One-year background in & Regression and time  &
 Chapters
1-8, 20-21, selected \\
~~~probability and statistics & ~~~series models &
~~~portions of technical supplements \\
One-year background in & Regression modeling  &
 Chapters
1-6, 11-13, 20-21, selected \\
~~~probability and statistics & &
~~~ portions of technical supplements \\
Background in statistics& Actuarial regression&
 Chapters
10-21, selected \\
~~~and linear regression&  ~~~models  &
~~~ portions of technical supplements \\
 \hline
\end{tabular}
\scalefont{1.1111}
\end{table}

In addition to these suggested courses, this book is designed for
supplemental reading for a time series course as well as a reference
book for industry analysts. My hope is that college students who use
the beginning parts of the book in their university course will find
the later chapters helpful in their industry positions. In this way
I hope to promote life-long learning!



\section*{Acknowledgements}

It is appropriate to begin the acknowledgement section by thanking
the students in the actuarial program here at the University of
Wisconsin; students are important partners in the knowledge creation
and dissemination business at universities. Through their questions
and feedback, I have learned a tremendous amount over the years. I
have also benefited from \emph{excellent} assistance from those who
have helped me pull together all the pieces for this book,
specifically, Missy Pinney, Peng Shi, Yunjie (Winnie) Sun and Ziyan
Xie.

I have enjoyed working with several former students and colleagues
on regression problems over the recent years, including Katrien
Antonio, Jie Gao, Paul Johnson, Margie Rosenberg, Jiafeng Sun, Emil
Valdez and Ping Wang. Their contributions are reflected indirectly
throughout the text. Because of my long association with the
University of Wisconsin-Madison, I am reluctant to go back further
in time and provide a longer list for fear of missing important
individuals. I have also been fortunate to have a more recent
association with the Insurance Services Office (ISO). Colleagues at
ISO have provided me with important insights into applications.
Through this text that features applications of regression into
actuarial and financial industry problems, I hope to encourage the
fostering of additional partnerships between academia and industry.



I am pleased to acknowledge detailed reviews that I have received
from colleagues Tim Welnetz and Margie Rosenberg. I also wish to
thank Bob Miller for permission to include our joint work on
designing effective graphs in Chapter 21. Bob has taught me a lot
about regression over the years.

Moreover, I am happy to acknowledge financial support through the
Assurant Health Professorship in Actuarial Science at the University
of Wisconsin-Madison.

Saving the most important for last, I thank my family for their
support. Ten thousand thanks to my mother Mary, brothers Randy, Guy
and Joe, my wife Deirdre and our sons Nathan and Adam.
